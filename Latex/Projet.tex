\documentclass[11pt]{article}

    \usepackage[breakable]{tcolorbox}
    \usepackage{parskip} % Stop auto-indenting (to mimic markdown behaviour)
    

    % Basic figure setup, for now with no caption control since it's done
    % automatically by Pandoc (which extracts ![](path) syntax from Markdown).
    \usepackage{graphicx}
    % Maintain compatibility with old templates. Remove in nbconvert 6.0
    \let\Oldincludegraphics\includegraphics
    % Ensure that by default, figures have no caption (until we provide a
    % proper Figure object with a Caption API and a way to capture that
    % in the conversion process - todo).
    \usepackage{caption}
    \DeclareCaptionFormat{nocaption}{}
    \captionsetup{format=nocaption,aboveskip=0pt,belowskip=0pt}

    \usepackage{float}
    \floatplacement{figure}{H} % forces figures to be placed at the correct location
    \usepackage{xcolor} % Allow colors to be defined
    \usepackage{enumerate} % Needed for markdown enumerations to work
    \usepackage{geometry} % Used to adjust the document margins
    \usepackage{amsmath} % Equations
    \usepackage{amssymb} % Equations
    \usepackage{textcomp} % defines textquotesingle
    % Hack from http://tex.stackexchange.com/a/47451/13684:
    \AtBeginDocument{%
        \def\PYZsq{\textquotesingle}% Upright quotes in Pygmentized code
    }
    \usepackage{upquote} % Upright quotes for verbatim code
    \usepackage{eurosym} % defines \euro

    \usepackage{iftex}
    \ifPDFTeX
        \usepackage[T1]{fontenc}
        \IfFileExists{alphabeta.sty}{
              \usepackage{alphabeta}
          }{
              \usepackage[mathletters]{ucs}
              \usepackage[utf8x]{inputenc}
          }
    \else
        \usepackage{fontspec}
        \usepackage{unicode-math}
    \fi

    \usepackage{fancyvrb} % verbatim replacement that allows latex
    \usepackage{grffile} % extends the file name processing of package graphics 
                         % to support a larger range
    \makeatletter % fix for old versions of grffile with XeLaTeX
    \@ifpackagelater{grffile}{2019/11/01}
    {
      % Do nothing on new versions
    }
    {
      \def\Gread@@xetex#1{%
        \IfFileExists{"\Gin@base".bb}%
        {\Gread@eps{\Gin@base.bb}}%
        {\Gread@@xetex@aux#1}%
      }
    }
    \makeatother
    \usepackage[Export]{adjustbox} % Used to constrain images to a maximum size
    \adjustboxset{max size={0.9\linewidth}{0.9\paperheight}}

    % The hyperref package gives us a pdf with properly built
    % internal navigation ('pdf bookmarks' for the table of contents,
    % internal cross-reference links, web links for URLs, etc.)
    \usepackage{hyperref}
    % The default LaTeX title has an obnoxious amount of whitespace. By default,
    % titling removes some of it. It also provides customization options.
    \usepackage{titling}
    \usepackage{longtable} % longtable support required by pandoc >1.10
    \usepackage{booktabs}  % table support for pandoc > 1.12.2
    \usepackage{array}     % table support for pandoc >= 2.11.3
    \usepackage{calc}      % table minipage width calculation for pandoc >= 2.11.1
    \usepackage[inline]{enumitem} % IRkernel/repr support (it uses the enumerate* environment)
    \usepackage[normalem]{ulem} % ulem is needed to support strikethroughs (\sout)
                                % normalem makes italics be italics, not underlines
    \usepackage{mathrsfs}
    

    
    % Colors for the hyperref package
    \definecolor{urlcolor}{rgb}{0,.145,.698}
    \definecolor{linkcolor}{rgb}{.71,0.21,0.01}
    \definecolor{citecolor}{rgb}{.12,.54,.11}

    % ANSI colors
    \definecolor{ansi-black}{HTML}{3E424D}
    \definecolor{ansi-black-intense}{HTML}{282C36}
    \definecolor{ansi-red}{HTML}{E75C58}
    \definecolor{ansi-red-intense}{HTML}{B22B31}
    \definecolor{ansi-green}{HTML}{00A250}
    \definecolor{ansi-green-intense}{HTML}{007427}
    \definecolor{ansi-yellow}{HTML}{DDB62B}
    \definecolor{ansi-yellow-intense}{HTML}{B27D12}
    \definecolor{ansi-blue}{HTML}{208FFB}
    \definecolor{ansi-blue-intense}{HTML}{0065CA}
    \definecolor{ansi-magenta}{HTML}{D160C4}
    \definecolor{ansi-magenta-intense}{HTML}{A03196}
    \definecolor{ansi-cyan}{HTML}{60C6C8}
    \definecolor{ansi-cyan-intense}{HTML}{258F8F}
    \definecolor{ansi-white}{HTML}{C5C1B4}
    \definecolor{ansi-white-intense}{HTML}{A1A6B2}
    \definecolor{ansi-default-inverse-fg}{HTML}{FFFFFF}
    \definecolor{ansi-default-inverse-bg}{HTML}{000000}

    % common color for the border for error outputs.
    \definecolor{outerrorbackground}{HTML}{FFDFDF}

    % commands and environments needed by pandoc snippets
    % extracted from the output of `pandoc -s`
    \providecommand{\tightlist}{%
      \setlength{\itemsep}{0pt}\setlength{\parskip}{0pt}}
    \DefineVerbatimEnvironment{Highlighting}{Verbatim}{commandchars=\\\{\}}
    % Add ',fontsize=\small' for more characters per line
    \newenvironment{Shaded}{}{}
    \newcommand{\KeywordTok}[1]{\textcolor[rgb]{0.00,0.44,0.13}{\textbf{{#1}}}}
    \newcommand{\DataTypeTok}[1]{\textcolor[rgb]{0.56,0.13,0.00}{{#1}}}
    \newcommand{\DecValTok}[1]{\textcolor[rgb]{0.25,0.63,0.44}{{#1}}}
    \newcommand{\BaseNTok}[1]{\textcolor[rgb]{0.25,0.63,0.44}{{#1}}}
    \newcommand{\FloatTok}[1]{\textcolor[rgb]{0.25,0.63,0.44}{{#1}}}
    \newcommand{\CharTok}[1]{\textcolor[rgb]{0.25,0.44,0.63}{{#1}}}
    \newcommand{\StringTok}[1]{\textcolor[rgb]{0.25,0.44,0.63}{{#1}}}
    \newcommand{\CommentTok}[1]{\textcolor[rgb]{0.38,0.63,0.69}{\textit{{#1}}}}
    \newcommand{\OtherTok}[1]{\textcolor[rgb]{0.00,0.44,0.13}{{#1}}}
    \newcommand{\AlertTok}[1]{\textcolor[rgb]{1.00,0.00,0.00}{\textbf{{#1}}}}
    \newcommand{\FunctionTok}[1]{\textcolor[rgb]{0.02,0.16,0.49}{{#1}}}
    \newcommand{\RegionMarkerTok}[1]{{#1}}
    \newcommand{\ErrorTok}[1]{\textcolor[rgb]{1.00,0.00,0.00}{\textbf{{#1}}}}
    \newcommand{\NormalTok}[1]{{#1}}
    
    % Additional commands for more recent versions of Pandoc
    \newcommand{\ConstantTok}[1]{\textcolor[rgb]{0.53,0.00,0.00}{{#1}}}
    \newcommand{\SpecialCharTok}[1]{\textcolor[rgb]{0.25,0.44,0.63}{{#1}}}
    \newcommand{\VerbatimStringTok}[1]{\textcolor[rgb]{0.25,0.44,0.63}{{#1}}}
    \newcommand{\SpecialStringTok}[1]{\textcolor[rgb]{0.73,0.40,0.53}{{#1}}}
    \newcommand{\ImportTok}[1]{{#1}}
    \newcommand{\DocumentationTok}[1]{\textcolor[rgb]{0.73,0.13,0.13}{\textit{{#1}}}}
    \newcommand{\AnnotationTok}[1]{\textcolor[rgb]{0.38,0.63,0.69}{\textbf{\textit{{#1}}}}}
    \newcommand{\CommentVarTok}[1]{\textcolor[rgb]{0.38,0.63,0.69}{\textbf{\textit{{#1}}}}}
    \newcommand{\VariableTok}[1]{\textcolor[rgb]{0.10,0.09,0.49}{{#1}}}
    \newcommand{\ControlFlowTok}[1]{\textcolor[rgb]{0.00,0.44,0.13}{\textbf{{#1}}}}
    \newcommand{\OperatorTok}[1]{\textcolor[rgb]{0.40,0.40,0.40}{{#1}}}
    \newcommand{\BuiltInTok}[1]{{#1}}
    \newcommand{\ExtensionTok}[1]{{#1}}
    \newcommand{\PreprocessorTok}[1]{\textcolor[rgb]{0.74,0.48,0.00}{{#1}}}
    \newcommand{\AttributeTok}[1]{\textcolor[rgb]{0.49,0.56,0.16}{{#1}}}
    \newcommand{\InformationTok}[1]{\textcolor[rgb]{0.38,0.63,0.69}{\textbf{\textit{{#1}}}}}
    \newcommand{\WarningTok}[1]{\textcolor[rgb]{0.38,0.63,0.69}{\textbf{\textit{{#1}}}}}
    
    
    % Define a nice break command that doesn't care if a line doesn't already
    % exist.
    \def\br{\hspace*{\fill} \\* }
    % Math Jax compatibility definitions
    \def\gt{>}
    \def\lt{<}
    \let\Oldtex\TeX
    \let\Oldlatex\LaTeX
    \renewcommand{\TeX}{\textrm{\Oldtex}}
    \renewcommand{\LaTeX}{\textrm{\Oldlatex}}
    % Document parameters
    % Document title
    \title{Projet}
    
    
    
    
    
% Pygments definitions
\makeatletter
\def\PY@reset{\let\PY@it=\relax \let\PY@bf=\relax%
    \let\PY@ul=\relax \let\PY@tc=\relax%
    \let\PY@bc=\relax \let\PY@ff=\relax}
\def\PY@tok#1{\csname PY@tok@#1\endcsname}
\def\PY@toks#1+{\ifx\relax#1\empty\else%
    \PY@tok{#1}\expandafter\PY@toks\fi}
\def\PY@do#1{\PY@bc{\PY@tc{\PY@ul{%
    \PY@it{\PY@bf{\PY@ff{#1}}}}}}}
\def\PY#1#2{\PY@reset\PY@toks#1+\relax+\PY@do{#2}}

\@namedef{PY@tok@w}{\def\PY@tc##1{\textcolor[rgb]{0.73,0.73,0.73}{##1}}}
\@namedef{PY@tok@c}{\let\PY@it=\textit\def\PY@tc##1{\textcolor[rgb]{0.24,0.48,0.48}{##1}}}
\@namedef{PY@tok@cp}{\def\PY@tc##1{\textcolor[rgb]{0.61,0.40,0.00}{##1}}}
\@namedef{PY@tok@k}{\let\PY@bf=\textbf\def\PY@tc##1{\textcolor[rgb]{0.00,0.50,0.00}{##1}}}
\@namedef{PY@tok@kp}{\def\PY@tc##1{\textcolor[rgb]{0.00,0.50,0.00}{##1}}}
\@namedef{PY@tok@kt}{\def\PY@tc##1{\textcolor[rgb]{0.69,0.00,0.25}{##1}}}
\@namedef{PY@tok@o}{\def\PY@tc##1{\textcolor[rgb]{0.40,0.40,0.40}{##1}}}
\@namedef{PY@tok@ow}{\let\PY@bf=\textbf\def\PY@tc##1{\textcolor[rgb]{0.67,0.13,1.00}{##1}}}
\@namedef{PY@tok@nb}{\def\PY@tc##1{\textcolor[rgb]{0.00,0.50,0.00}{##1}}}
\@namedef{PY@tok@nf}{\def\PY@tc##1{\textcolor[rgb]{0.00,0.00,1.00}{##1}}}
\@namedef{PY@tok@nc}{\let\PY@bf=\textbf\def\PY@tc##1{\textcolor[rgb]{0.00,0.00,1.00}{##1}}}
\@namedef{PY@tok@nn}{\let\PY@bf=\textbf\def\PY@tc##1{\textcolor[rgb]{0.00,0.00,1.00}{##1}}}
\@namedef{PY@tok@ne}{\let\PY@bf=\textbf\def\PY@tc##1{\textcolor[rgb]{0.80,0.25,0.22}{##1}}}
\@namedef{PY@tok@nv}{\def\PY@tc##1{\textcolor[rgb]{0.10,0.09,0.49}{##1}}}
\@namedef{PY@tok@no}{\def\PY@tc##1{\textcolor[rgb]{0.53,0.00,0.00}{##1}}}
\@namedef{PY@tok@nl}{\def\PY@tc##1{\textcolor[rgb]{0.46,0.46,0.00}{##1}}}
\@namedef{PY@tok@ni}{\let\PY@bf=\textbf\def\PY@tc##1{\textcolor[rgb]{0.44,0.44,0.44}{##1}}}
\@namedef{PY@tok@na}{\def\PY@tc##1{\textcolor[rgb]{0.41,0.47,0.13}{##1}}}
\@namedef{PY@tok@nt}{\let\PY@bf=\textbf\def\PY@tc##1{\textcolor[rgb]{0.00,0.50,0.00}{##1}}}
\@namedef{PY@tok@nd}{\def\PY@tc##1{\textcolor[rgb]{0.67,0.13,1.00}{##1}}}
\@namedef{PY@tok@s}{\def\PY@tc##1{\textcolor[rgb]{0.73,0.13,0.13}{##1}}}
\@namedef{PY@tok@sd}{\let\PY@it=\textit\def\PY@tc##1{\textcolor[rgb]{0.73,0.13,0.13}{##1}}}
\@namedef{PY@tok@si}{\let\PY@bf=\textbf\def\PY@tc##1{\textcolor[rgb]{0.64,0.35,0.47}{##1}}}
\@namedef{PY@tok@se}{\let\PY@bf=\textbf\def\PY@tc##1{\textcolor[rgb]{0.67,0.36,0.12}{##1}}}
\@namedef{PY@tok@sr}{\def\PY@tc##1{\textcolor[rgb]{0.64,0.35,0.47}{##1}}}
\@namedef{PY@tok@ss}{\def\PY@tc##1{\textcolor[rgb]{0.10,0.09,0.49}{##1}}}
\@namedef{PY@tok@sx}{\def\PY@tc##1{\textcolor[rgb]{0.00,0.50,0.00}{##1}}}
\@namedef{PY@tok@m}{\def\PY@tc##1{\textcolor[rgb]{0.40,0.40,0.40}{##1}}}
\@namedef{PY@tok@gh}{\let\PY@bf=\textbf\def\PY@tc##1{\textcolor[rgb]{0.00,0.00,0.50}{##1}}}
\@namedef{PY@tok@gu}{\let\PY@bf=\textbf\def\PY@tc##1{\textcolor[rgb]{0.50,0.00,0.50}{##1}}}
\@namedef{PY@tok@gd}{\def\PY@tc##1{\textcolor[rgb]{0.63,0.00,0.00}{##1}}}
\@namedef{PY@tok@gi}{\def\PY@tc##1{\textcolor[rgb]{0.00,0.52,0.00}{##1}}}
\@namedef{PY@tok@gr}{\def\PY@tc##1{\textcolor[rgb]{0.89,0.00,0.00}{##1}}}
\@namedef{PY@tok@ge}{\let\PY@it=\textit}
\@namedef{PY@tok@gs}{\let\PY@bf=\textbf}
\@namedef{PY@tok@gp}{\let\PY@bf=\textbf\def\PY@tc##1{\textcolor[rgb]{0.00,0.00,0.50}{##1}}}
\@namedef{PY@tok@go}{\def\PY@tc##1{\textcolor[rgb]{0.44,0.44,0.44}{##1}}}
\@namedef{PY@tok@gt}{\def\PY@tc##1{\textcolor[rgb]{0.00,0.27,0.87}{##1}}}
\@namedef{PY@tok@err}{\def\PY@bc##1{{\setlength{\fboxsep}{\string -\fboxrule}\fcolorbox[rgb]{1.00,0.00,0.00}{1,1,1}{\strut ##1}}}}
\@namedef{PY@tok@kc}{\let\PY@bf=\textbf\def\PY@tc##1{\textcolor[rgb]{0.00,0.50,0.00}{##1}}}
\@namedef{PY@tok@kd}{\let\PY@bf=\textbf\def\PY@tc##1{\textcolor[rgb]{0.00,0.50,0.00}{##1}}}
\@namedef{PY@tok@kn}{\let\PY@bf=\textbf\def\PY@tc##1{\textcolor[rgb]{0.00,0.50,0.00}{##1}}}
\@namedef{PY@tok@kr}{\let\PY@bf=\textbf\def\PY@tc##1{\textcolor[rgb]{0.00,0.50,0.00}{##1}}}
\@namedef{PY@tok@bp}{\def\PY@tc##1{\textcolor[rgb]{0.00,0.50,0.00}{##1}}}
\@namedef{PY@tok@fm}{\def\PY@tc##1{\textcolor[rgb]{0.00,0.00,1.00}{##1}}}
\@namedef{PY@tok@vc}{\def\PY@tc##1{\textcolor[rgb]{0.10,0.09,0.49}{##1}}}
\@namedef{PY@tok@vg}{\def\PY@tc##1{\textcolor[rgb]{0.10,0.09,0.49}{##1}}}
\@namedef{PY@tok@vi}{\def\PY@tc##1{\textcolor[rgb]{0.10,0.09,0.49}{##1}}}
\@namedef{PY@tok@vm}{\def\PY@tc##1{\textcolor[rgb]{0.10,0.09,0.49}{##1}}}
\@namedef{PY@tok@sa}{\def\PY@tc##1{\textcolor[rgb]{0.73,0.13,0.13}{##1}}}
\@namedef{PY@tok@sb}{\def\PY@tc##1{\textcolor[rgb]{0.73,0.13,0.13}{##1}}}
\@namedef{PY@tok@sc}{\def\PY@tc##1{\textcolor[rgb]{0.73,0.13,0.13}{##1}}}
\@namedef{PY@tok@dl}{\def\PY@tc##1{\textcolor[rgb]{0.73,0.13,0.13}{##1}}}
\@namedef{PY@tok@s2}{\def\PY@tc##1{\textcolor[rgb]{0.73,0.13,0.13}{##1}}}
\@namedef{PY@tok@sh}{\def\PY@tc##1{\textcolor[rgb]{0.73,0.13,0.13}{##1}}}
\@namedef{PY@tok@s1}{\def\PY@tc##1{\textcolor[rgb]{0.73,0.13,0.13}{##1}}}
\@namedef{PY@tok@mb}{\def\PY@tc##1{\textcolor[rgb]{0.40,0.40,0.40}{##1}}}
\@namedef{PY@tok@mf}{\def\PY@tc##1{\textcolor[rgb]{0.40,0.40,0.40}{##1}}}
\@namedef{PY@tok@mh}{\def\PY@tc##1{\textcolor[rgb]{0.40,0.40,0.40}{##1}}}
\@namedef{PY@tok@mi}{\def\PY@tc##1{\textcolor[rgb]{0.40,0.40,0.40}{##1}}}
\@namedef{PY@tok@il}{\def\PY@tc##1{\textcolor[rgb]{0.40,0.40,0.40}{##1}}}
\@namedef{PY@tok@mo}{\def\PY@tc##1{\textcolor[rgb]{0.40,0.40,0.40}{##1}}}
\@namedef{PY@tok@ch}{\let\PY@it=\textit\def\PY@tc##1{\textcolor[rgb]{0.24,0.48,0.48}{##1}}}
\@namedef{PY@tok@cm}{\let\PY@it=\textit\def\PY@tc##1{\textcolor[rgb]{0.24,0.48,0.48}{##1}}}
\@namedef{PY@tok@cpf}{\let\PY@it=\textit\def\PY@tc##1{\textcolor[rgb]{0.24,0.48,0.48}{##1}}}
\@namedef{PY@tok@c1}{\let\PY@it=\textit\def\PY@tc##1{\textcolor[rgb]{0.24,0.48,0.48}{##1}}}
\@namedef{PY@tok@cs}{\let\PY@it=\textit\def\PY@tc##1{\textcolor[rgb]{0.24,0.48,0.48}{##1}}}

\def\PYZbs{\char`\\}
\def\PYZus{\char`\_}
\def\PYZob{\char`\{}
\def\PYZcb{\char`\}}
\def\PYZca{\char`\^}
\def\PYZam{\char`\&}
\def\PYZlt{\char`\<}
\def\PYZgt{\char`\>}
\def\PYZsh{\char`\#}
\def\PYZpc{\char`\%}
\def\PYZdl{\char`\$}
\def\PYZhy{\char`\-}
\def\PYZsq{\char`\'}
\def\PYZdq{\char`\"}
\def\PYZti{\char`\~}
% for compatibility with earlier versions
\def\PYZat{@}
\def\PYZlb{[}
\def\PYZrb{]}
\makeatother


    % For linebreaks inside Verbatim environment from package fancyvrb. 
    \makeatletter
        \newbox\Wrappedcontinuationbox 
        \newbox\Wrappedvisiblespacebox 
        \newcommand*\Wrappedvisiblespace {\textcolor{red}{\textvisiblespace}} 
        \newcommand*\Wrappedcontinuationsymbol {\textcolor{red}{\llap{\tiny$\m@th\hookrightarrow$}}} 
        \newcommand*\Wrappedcontinuationindent {3ex } 
        \newcommand*\Wrappedafterbreak {\kern\Wrappedcontinuationindent\copy\Wrappedcontinuationbox} 
        % Take advantage of the already applied Pygments mark-up to insert 
        % potential linebreaks for TeX processing. 
        %        {, <, #, %, $, ' and ": go to next line. 
        %        _, }, ^, &, >, - and ~: stay at end of broken line. 
        % Use of \textquotesingle for straight quote. 
        \newcommand*\Wrappedbreaksatspecials {% 
            \def\PYGZus{\discretionary{\char`\_}{\Wrappedafterbreak}{\char`\_}}% 
            \def\PYGZob{\discretionary{}{\Wrappedafterbreak\char`\{}{\char`\{}}% 
            \def\PYGZcb{\discretionary{\char`\}}{\Wrappedafterbreak}{\char`\}}}% 
            \def\PYGZca{\discretionary{\char`\^}{\Wrappedafterbreak}{\char`\^}}% 
            \def\PYGZam{\discretionary{\char`\&}{\Wrappedafterbreak}{\char`\&}}% 
            \def\PYGZlt{\discretionary{}{\Wrappedafterbreak\char`\<}{\char`\<}}% 
            \def\PYGZgt{\discretionary{\char`\>}{\Wrappedafterbreak}{\char`\>}}% 
            \def\PYGZsh{\discretionary{}{\Wrappedafterbreak\char`\#}{\char`\#}}% 
            \def\PYGZpc{\discretionary{}{\Wrappedafterbreak\char`\%}{\char`\%}}% 
            \def\PYGZdl{\discretionary{}{\Wrappedafterbreak\char`\$}{\char`\$}}% 
            \def\PYGZhy{\discretionary{\char`\-}{\Wrappedafterbreak}{\char`\-}}% 
            \def\PYGZsq{\discretionary{}{\Wrappedafterbreak\textquotesingle}{\textquotesingle}}% 
            \def\PYGZdq{\discretionary{}{\Wrappedafterbreak\char`\"}{\char`\"}}% 
            \def\PYGZti{\discretionary{\char`\~}{\Wrappedafterbreak}{\char`\~}}% 
        } 
        % Some characters . , ; ? ! / are not pygmentized. 
        % This macro makes them "active" and they will insert potential linebreaks 
        \newcommand*\Wrappedbreaksatpunct {% 
            \lccode`\~`\.\lowercase{\def~}{\discretionary{\hbox{\char`\.}}{\Wrappedafterbreak}{\hbox{\char`\.}}}% 
            \lccode`\~`\,\lowercase{\def~}{\discretionary{\hbox{\char`\,}}{\Wrappedafterbreak}{\hbox{\char`\,}}}% 
            \lccode`\~`\;\lowercase{\def~}{\discretionary{\hbox{\char`\;}}{\Wrappedafterbreak}{\hbox{\char`\;}}}% 
            \lccode`\~`\:\lowercase{\def~}{\discretionary{\hbox{\char`\:}}{\Wrappedafterbreak}{\hbox{\char`\:}}}% 
            \lccode`\~`\?\lowercase{\def~}{\discretionary{\hbox{\char`\?}}{\Wrappedafterbreak}{\hbox{\char`\?}}}% 
            \lccode`\~`\!\lowercase{\def~}{\discretionary{\hbox{\char`\!}}{\Wrappedafterbreak}{\hbox{\char`\!}}}% 
            \lccode`\~`\/\lowercase{\def~}{\discretionary{\hbox{\char`\/}}{\Wrappedafterbreak}{\hbox{\char`\/}}}% 
            \catcode`\.\active
            \catcode`\,\active 
            \catcode`\;\active
            \catcode`\:\active
            \catcode`\?\active
            \catcode`\!\active
            \catcode`\/\active 
            \lccode`\~`\~ 	
        }
    \makeatother

    \let\OriginalVerbatim=\Verbatim
    \makeatletter
    \renewcommand{\Verbatim}[1][1]{%
        %\parskip\z@skip
        \sbox\Wrappedcontinuationbox {\Wrappedcontinuationsymbol}%
        \sbox\Wrappedvisiblespacebox {\FV@SetupFont\Wrappedvisiblespace}%
        \def\FancyVerbFormatLine ##1{\hsize\linewidth
            \vtop{\raggedright\hyphenpenalty\z@\exhyphenpenalty\z@
                \doublehyphendemerits\z@\finalhyphendemerits\z@
                \strut ##1\strut}%
        }%
        % If the linebreak is at a space, the latter will be displayed as visible
        % space at end of first line, and a continuation symbol starts next line.
        % Stretch/shrink are however usually zero for typewriter font.
        \def\FV@Space {%
            \nobreak\hskip\z@ plus\fontdimen3\font minus\fontdimen4\font
            \discretionary{\copy\Wrappedvisiblespacebox}{\Wrappedafterbreak}
            {\kern\fontdimen2\font}%
        }%
        
        % Allow breaks at special characters using \PYG... macros.
        \Wrappedbreaksatspecials
        % Breaks at punctuation characters . , ; ? ! and / need catcode=\active 	
        \OriginalVerbatim[#1,codes*=\Wrappedbreaksatpunct]%
    }
    \makeatother

    % Exact colors from NB
    \definecolor{incolor}{HTML}{303F9F}
    \definecolor{outcolor}{HTML}{D84315}
    \definecolor{cellborder}{HTML}{CFCFCF}
    \definecolor{cellbackground}{HTML}{F7F7F7}
    
    % prompt
    \makeatletter
    \newcommand{\boxspacing}{\kern\kvtcb@left@rule\kern\kvtcb@boxsep}
    \makeatother
    \newcommand{\prompt}[4]{
        {\ttfamily\llap{{\color{#2}[#3]:\hspace{3pt}#4}}\vspace{-\baselineskip}}
    }
    

    
    % Prevent overflowing lines due to hard-to-break entities
    \sloppy 
    % Setup hyperref package
    \hypersetup{
      breaklinks=true,  % so long urls are correctly broken across lines
      colorlinks=true,
      urlcolor=urlcolor,
      linkcolor=linkcolor,
      citecolor=citecolor,
      }
    % Slightly bigger margins than the latex defaults
    
    \geometry{verbose,tmargin=1in,bmargin=1in,lmargin=1in,rmargin=1in}
    
    

\begin{document}
    
    \maketitle
    
    

    
    \textbf{\emph{Analyse, classification et prédiction de consommation
d'électricité en France par des techniques de machine learning}}

    \hypertarget{introduction}{%
\section{Introduction}\label{introduction}}

L'analyse, la classification et la prédiction de la consommation
d'électricité sont des sujets importants pour comprendre et améliorer
l'efficacité énergétique. Les techniques de machine learning peuvent
être utilisées pour extraire des informations utiles à partir de données
de consommation d'électricité, ainsi que pour prédire les tendances
futures de consommation. Cette analyse peut aider les entreprises de
l'industrie électrique à mieux comprendre les comportements de
consommation de leurs clients, ainsi qu'à développer des stratégies pour
améliorer l'efficacité énergétique à tous les niveaux

    La production d'électricité en France varie en fonction de différents
facteurs, tels que la demande, les conditions météorologiques, les
contraintes techniques et les politiques énergétiques. La France importe
également une partie de son électricité pour répondre à sa demande.

    Nous verrons comment les algorithmes d'apprentissage automatique peuvent
être utilisés pour extraire des informations précieuses à partir de
données de consommation d'électricité, telles que les modèles de
consommation, les tendances de consommation et les prévisions de
consommation future

    \begin{tcolorbox}[breakable, size=fbox, boxrule=1pt, pad at break*=1mm,colback=cellbackground, colframe=cellborder]
\prompt{In}{incolor}{1}{\boxspacing}
\begin{Verbatim}[commandchars=\\\{\}]
\PY{c+c1}{\PYZsh{} \PYZhy{}*\PYZhy{} coding: utf\PYZhy{}8 \PYZhy{}*\PYZhy{}}
\PY{c+c1}{\PYZsh{}importer les librairies dont nous aurons besoin pour ce projet}
\PY{c+c1}{\PYZsh{}Matplotlib est la librairie qui permet de visualiser nos Datasets, nos fonctions, nos résultats sous forme de graphiques, courbes et nuages de points.}
\PY{k+kn}{import} \PY{n+nn}{matplotlib}\PY{n+nn}{.}\PY{n+nn}{pyplot} \PY{k}{as} \PY{n+nn}{plt}
\PY{c+c1}{\PYZsh{}pandas pour l\PYZsq{}extraction des données et le traitement}
\PY{k+kn}{import} \PY{n+nn}{pandas} \PY{k}{as} \PY{n+nn}{pd}
\PY{c+c1}{\PYZsh{}Numpy pour manipuler notre Dataset en tant que matrice}
\PY{k+kn}{import} \PY{n+nn}{numpy} \PY{k}{as} \PY{n+nn}{np}
\PY{k+kn}{import} \PY{n+nn}{statsmodels}\PY{n+nn}{.}\PY{n+nn}{api} \PY{k}{as} \PY{n+nn}{sm}

\PY{c+c1}{\PYZsh{}our l\PYZsq{}affichage des donnée tableau}
\PY{k+kn}{from} \PY{n+nn}{IPython}\PY{n+nn}{.}\PY{n+nn}{display} \PY{k+kn}{import} \PY{n}{display}

\PY{c+c1}{\PYZsh{}Calculez les quantiles d\PYZsq{}un diagramme de probabilité et affichez éventuellement le diagramme}
\PY{k+kn}{from} \PY{n+nn}{statsmodels}\PY{n+nn}{.}\PY{n+nn}{graphics}\PY{n+nn}{.}\PY{n+nn}{gofplots} \PY{k+kn}{import} \PY{n}{ProbPlot}

\PY{c+c1}{\PYZsh{}pour ignorer tous les avertissements qui pourraient être générés pendant l\PYZsq{}exécution du code}
\PY{k+kn}{import} \PY{n+nn}{warnings}                                  
\PY{n}{warnings}\PY{o}{.}\PY{n}{filterwarnings}\PY{p}{(}\PY{l+s+s1}{\PYZsq{}}\PY{l+s+s1}{ignore}\PY{l+s+s1}{\PYZsq{}}\PY{p}{)}

\PY{k+kn}{from} \PY{n+nn}{matplotlib}\PY{n+nn}{.}\PY{n+nn}{pyplot} \PY{k+kn}{import} \PY{n}{figure}
\end{Verbatim}
\end{tcolorbox}

    \hypertarget{nettoyage-analyse}{%
\subsection{Nettoyage, Analyse}\label{nettoyage-analyse}}

Analyse, classification et prédiction de consommation d'électricité par
des techniques de machine learning qui s'inscrit dans le contexte de
l'analyse des données (Data Analytics) \emph{Source de nos données}\\
\href{https://www.rte-france.com/eco2mix/telecharger-les-indicateurs}{Données
en énergie (consolidées et définitives) RTE FRANCE}\\
Elles concernent la France et les 12 régions administratives.\\
\href{https://cegibat.grdf.fr/simulateur/calcul-dju}{Calcul des degrés
jours unifiés (DJU) CEGIBAT}\\
Cet outil, réalisé en partenariat avec Météo France, permet de calculer
les degrés jour (DJ ou DJU) chauffage ou climatisation sur une période,
une station météo et un seuil de température donnés.\\

\emph{Les données de l'energie sont exprimées en Gigawatt (GWh).}~

    \begin{tcolorbox}[breakable, size=fbox, boxrule=1pt, pad at break*=1mm,colback=cellbackground, colframe=cellborder]
\prompt{In}{incolor}{2}{\boxspacing}
\begin{Verbatim}[commandchars=\\\{\}]
\PY{c+c1}{\PYZsh{}Traitement du jeu de donnée}

\PY{c+c1}{\PYZsh{}Chargement du dataset \PYZsq{}new\PYZus{}data \PYZhy{} new\PYZus{}data.csv\PYZsq{}}
\PY{n}{data\PYZus{}coso\PYZus{}prod}\PY{o}{=}\PY{n}{pd}\PY{o}{.}\PY{n}{read\PYZus{}csv}\PY{p}{(}\PY{l+s+s2}{\PYZdq{}}\PY{l+s+s2}{new\PYZus{}data \PYZhy{} new\PYZus{}data.csv}\PY{l+s+s2}{\PYZdq{}}\PY{p}{)}
\PY{c+c1}{\PYZsh{}afficher les 10 prémière ligne des données importées}
\PY{n}{data\PYZus{}coso\PYZus{}prod}\PY{o}{.}\PY{n}{iloc}\PY{p}{[}\PY{l+m+mi}{35}\PY{p}{:}\PY{p}{]}\PY{o}{.}\PY{n}{head}\PY{p}{(}\PY{l+m+mi}{15}\PY{p}{)}
\end{Verbatim}
\end{tcolorbox}

            \begin{tcolorbox}[breakable, size=fbox, boxrule=.5pt, pad at break*=1mm, opacityfill=0]
\prompt{Out}{outcolor}{2}{\boxspacing}
\begin{Verbatim}[commandchars=\\\{\}]
       Mois              Qualité               Territoire  Production totale  \textbackslash{}
35  2012-12  Données définitives                   France              54715
36  2013-01  Données définitives                Grand-Est               9924
37  2013-01  Données définitives       Nouvelle-Aquitaine               5644
38  2013-01  Données définitives     Auvergne-Rhône-Alpes              13119
39  2013-01  Données définitives  Bourgogne-Franche-Comté                403
40  2013-01  Données définitives                 Bretagne                246
41  2013-01  Données définitives      Centre-Val de Loire               7965
42  2013-01  Données définitives                   France              57739
43  2013-01  Données définitives            Ile-de-France                870
44  2013-01  Données définitives                Occitanie               3450
45  2013-01  Données définitives                Normandie               8304
46  2013-01  Données définitives          Hauts-de-France               4786
47  2013-01  Données définitives                     PACA               1941
48  2013-01  Données définitives         Pays-de-la-Loire                933
49  2013-02  Données définitives                Grand-Est               9057

    Production nucléaire  Production thermique totale  \textbackslash{}
35               39912.0                       5319.0
36                7258.0                       1508.0
37                4801.0                        187.0
38                9641.0                        278.0
39                   NaN                        244.0
40                   NaN                         49.0
41                7717.0                         70.0
42               42406.0                       6301.0
43                   NaN                        753.0
44                1853.0                         76.0
45                7473.0                        677.0
46                3660.0                        797.0
47                   NaN                        759.0
48                   NaN                        817.0
49                6086.0                       1863.0

    Production thermique charbon  Production thermique fioul  \textbackslash{}
35                        1714.0                       548.0
36                         928.0                         8.0
37                           NaN                         0.0
38                           0.0                        49.0
39                          75.0                         0.0
40                           NaN                         1.0
41                           NaN                         2.0
42                        2566.0                       393.0
43                         195.0                        37.0
44                           NaN                        10.0
45                         392.0                        47.0
46                         118.0                         0.0
47                         254.0                       128.0
48                         600.0                        22.0
49                        1087.0                         8.0

    Production thermique gaz  Production hydraulique  {\ldots}  \textbackslash{}
35                    3057.0                  6521.0  {\ldots}
36                     570.0                   814.0  {\ldots}
37                     186.0                   465.0  {\ldots}
38                     227.0                  3047.0  {\ldots}
39                     168.0                   121.0  {\ldots}
40                      48.0                    56.0  {\ldots}
41                      67.0                    24.0  {\ldots}
42                    3342.0                  6906.0  {\ldots}
43                     520.0                     0.0  {\ldots}
44                      65.0                  1214.0  {\ldots}
45                     238.0                    15.0  {\ldots}
46                     678.0                     1.0  {\ldots}
47                     376.0                  1079.0  {\ldots}
48                     193.0                     1.0  {\ldots}
49                     768.0                   708.0  {\ldots}

    Consommation totale  Solde exportateur  Echanges export  Echanges import  \textbackslash{}
35                49602             4367.0           6716.0           2349.0
36                    0                0.0              NaN              NaN
37                    0                0.0              NaN              NaN
38                    0                0.0              NaN              NaN
39                    0                0.0              NaN              NaN
40                    0                0.0              NaN              NaN
41                    0                NaN              NaN              NaN
42                53619             3484.0           6604.0           3120.0
43                    0                NaN              NaN              NaN
44                    0                NaN              NaN              NaN
45                    0                NaN              NaN              NaN
46                    0                NaN              NaN              NaN
47                    0                NaN              NaN              NaN
48                    0                NaN              NaN              NaN
49                    0                0.0              NaN              NaN

    Echanges avec le Royaume-Uni  Echanges avec l'Espagne  \textbackslash{}
35                         529.0                    -30.0
36                           NaN                      NaN
37                           NaN                      NaN
38                           NaN                      NaN
39                           NaN                      NaN
40                           NaN                      NaN
41                           NaN                      NaN
42                         493.0                     91.0
43                           NaN                      NaN
44                           NaN                      NaN
45                           NaN                      NaN
46                           NaN                      NaN
47                           NaN                      NaN
48                           NaN                      NaN
49                           NaN                      NaN

    Echanges avec l'Italie  Echanges avec la Suisse  \textbackslash{}
35                  1292.0                   1718.0
36                     NaN                      NaN
37                     NaN                      NaN
38                     NaN                      NaN
39                     NaN                      NaN
40                     NaN                      NaN
41                     NaN                      NaN
42                  1499.0                   1666.0
43                     NaN                      NaN
44                     NaN                      NaN
45                     NaN                      NaN
46                     NaN                      NaN
47                     NaN                      NaN
48                     NaN                      NaN
49                     NaN                      NaN

    Echanges avec l'Allemagne et la Belgique  Unnamed: 22
35                                     859.0          NaN
36                                       NaN          NaN
37                                       NaN          NaN
38                                       NaN          NaN
39                                       NaN          NaN
40                                       NaN          NaN
41                                       NaN          NaN
42                                    -265.0          NaN
43                                       NaN          NaN
44                                       NaN          NaN
45                                       NaN          NaN
46                                       NaN          NaN
47                                       NaN          NaN
48                                       NaN          NaN
49                                       NaN          NaN

[15 rows x 23 columns]
\end{Verbatim}
\end{tcolorbox}
        
    \hypertarget{collecte-et-pruxe9-traitement-de-donnuxe9es-de-consommation}{%
\section{Collecte et pré-traitement de données de
consommation}\label{collecte-et-pruxe9-traitement-de-donnuxe9es-de-consommation}}

Création du dataset de consommation mensuelle d'électricité en France

    \begin{tcolorbox}[breakable, size=fbox, boxrule=1pt, pad at break*=1mm,colback=cellbackground, colframe=cellborder]
\prompt{In}{incolor}{3}{\boxspacing}
\begin{Verbatim}[commandchars=\\\{\}]
\PY{c+c1}{\PYZsh{}On prend juste les données de la france}
\PY{c+c1}{\PYZsh{}Copier dans consommation}
\PY{n}{consommation}\PY{o}{=}\PY{n}{data\PYZus{}coso\PYZus{}prod}\PY{o}{.}\PY{n}{copy}\PY{p}{(}\PY{p}{)}
\PY{c+c1}{\PYZsh{}Trier et extraire seulement ce qui nous intéresse}
\PY{n}{consommation}\PY{o}{=}\PY{n}{consommation}\PY{p}{[}\PY{p}{[}\PY{l+s+s1}{\PYZsq{}}\PY{l+s+s1}{Mois}\PY{l+s+s1}{\PYZsq{}}\PY{p}{,} \PY{l+s+s1}{\PYZsq{}}\PY{l+s+s1}{Territoire}\PY{l+s+s1}{\PYZsq{}}\PY{p}{,} \PY{l+s+s1}{\PYZsq{}}\PY{l+s+s1}{Consommation totale}\PY{l+s+s1}{\PYZsq{}}\PY{p}{]}\PY{p}{]}
\PY{c+c1}{\PYZsh{}On récupère les information où Territoire=France et on laisse les données des régions}
\PY{n}{consommation}\PY{o}{=}\PY{n}{consommation}\PY{p}{[}\PY{n}{consommation}\PY{p}{[}\PY{l+s+s1}{\PYZsq{}}\PY{l+s+s1}{Territoire}\PY{l+s+s1}{\PYZsq{}}\PY{p}{]}\PY{o}{==}\PY{l+s+s1}{\PYZsq{}}\PY{l+s+s1}{France}\PY{l+s+s1}{\PYZsq{}}\PY{p}{]}
\PY{c+c1}{\PYZsh{}Eliminer les données manquantes}
\PY{n}{consommation}\PY{o}{.}\PY{n}{drop}\PY{p}{(}\PY{l+s+s1}{\PYZsq{}}\PY{l+s+s1}{Territoire}\PY{l+s+s1}{\PYZsq{}}\PY{p}{,} \PY{n}{axis}\PY{o}{=}\PY{l+m+mi}{1}\PY{p}{,} \PY{n}{inplace}\PY{o}{=}\PY{k+kc}{True}\PY{p}{)}
\PY{c+c1}{\PYZsh{}renommer les colonnes Mois et Consommation Totale}
\PY{n}{consommation}\PY{o}{.}\PY{n}{rename}\PY{p}{(}\PY{n}{columns}\PY{o}{=}\PY{p}{\PYZob{}}\PY{l+s+s1}{\PYZsq{}}\PY{l+s+s1}{Mois}\PY{l+s+s1}{\PYZsq{}}\PY{p}{:} \PY{l+s+s1}{\PYZsq{}}\PY{l+s+s1}{mois}\PY{l+s+s1}{\PYZsq{}}\PY{p}{,} \PY{l+s+s1}{\PYZsq{}}\PY{l+s+s1}{Consommation totale}\PY{l+s+s1}{\PYZsq{}}\PY{p}{:} \PY{l+s+s1}{\PYZsq{}}\PY{l+s+s1}{consommationT}\PY{l+s+s1}{\PYZsq{}}\PY{p}{\PYZcb{}}\PY{p}{,} \PY{n}{inplace}\PY{o}{=}\PY{k+kc}{True}\PY{p}{)}
\PY{c+c1}{\PYZsh{}Aperçu des données mensuelles de consommation totale d\PYZsq{}électricité de notre copy consommation}
\PY{n}{display}\PY{p}{(}\PY{n}{consommation}\PY{o}{.}\PY{n}{head}\PY{p}{(}\PY{p}{)}\PY{p}{)}
\PY{c+c1}{\PYZsh{}nobre de ligne et de colonne}
\PY{n}{display}\PY{p}{(}\PY{n}{consommation}\PY{o}{.}\PY{n}{shape}\PY{p}{)}
\end{Verbatim}
\end{tcolorbox}

    
    \begin{Verbatim}[commandchars=\\\{\}]
      mois  consommationT
0  2010-01          56342
1  2010-02          48698
2  2010-03          48294
3  2010-04          38637
4  2010-05          37284
    \end{Verbatim}

    
    
    \begin{Verbatim}[commandchars=\\\{\}]
(143, 2)
    \end{Verbatim}

    
    \begin{tcolorbox}[breakable, size=fbox, boxrule=1pt, pad at break*=1mm,colback=cellbackground, colframe=cellborder]
\prompt{In}{incolor}{4}{\boxspacing}
\begin{Verbatim}[commandchars=\\\{\}]
\PY{c+c1}{\PYZsh{}Triez par les valeurs le long de chaque axe mois par ordre croissant}
\PY{n}{consommation}\PY{o}{.}\PY{n}{sort\PYZus{}values}\PY{p}{(}\PY{n}{by}\PY{o}{=}\PY{l+s+s1}{\PYZsq{}}\PY{l+s+s1}{mois}\PY{l+s+s1}{\PYZsq{}}\PY{p}{,} \PY{n}{ascending}\PY{o}{=}\PY{k+kc}{True}\PY{p}{)}\PY{o}{.}\PY{n}{head}\PY{p}{(}\PY{p}{)}
\end{Verbatim}
\end{tcolorbox}

            \begin{tcolorbox}[breakable, size=fbox, boxrule=.5pt, pad at break*=1mm, opacityfill=0]
\prompt{Out}{outcolor}{4}{\boxspacing}
\begin{Verbatim}[commandchars=\\\{\}]
      mois  consommationT
0  2010-01          56342
1  2010-02          48698
2  2010-03          48294
3  2010-04          38637
4  2010-05          37284
\end{Verbatim}
\end{tcolorbox}
        
    \begin{tcolorbox}[breakable, size=fbox, boxrule=1pt, pad at break*=1mm,colback=cellbackground, colframe=cellborder]
\prompt{In}{incolor}{5}{\boxspacing}
\begin{Verbatim}[commandchars=\\\{\}]
\PY{c+c1}{\PYZsh{}Changement du type de Series en datetime}
\PY{c+c1}{\PYZsh{}Changement d\PYZsq{}index pour que la série temporelle soit reconnue}
\PY{n}{consommation}\PY{p}{[}\PY{l+s+s1}{\PYZsq{}}\PY{l+s+s1}{mois}\PY{l+s+s1}{\PYZsq{}}\PY{p}{]} \PY{o}{=} \PY{n}{pd}\PY{o}{.}\PY{n}{to\PYZus{}datetime}\PY{p}{(}\PY{n}{consommation}\PY{p}{[}\PY{l+s+s1}{\PYZsq{}}\PY{l+s+s1}{mois}\PY{l+s+s1}{\PYZsq{}}\PY{p}{]}\PY{p}{)}
\PY{n}{consommation}\PY{o}{.}\PY{n}{set\PYZus{}index}\PY{p}{(}\PY{l+s+s1}{\PYZsq{}}\PY{l+s+s1}{mois}\PY{l+s+s1}{\PYZsq{}}\PY{p}{,} \PY{n}{inplace}\PY{o}{=}\PY{k+kc}{True}\PY{p}{)}
\PY{c+c1}{\PYZsh{}Visualisation des données selon l\PYZsq{}indice temporel mois/année}
\PY{c+c1}{\PYZsh{}print(consommation)}
\PY{n}{consommation}\PY{o}{.}\PY{n}{head}\PY{p}{(}\PY{p}{)}
\end{Verbatim}
\end{tcolorbox}

            \begin{tcolorbox}[breakable, size=fbox, boxrule=.5pt, pad at break*=1mm, opacityfill=0]
\prompt{Out}{outcolor}{5}{\boxspacing}
\begin{Verbatim}[commandchars=\\\{\}]
            consommationT
mois
2010-01-01          56342
2010-02-01          48698
2010-03-01          48294
2010-04-01          38637
2010-05-01          37284
\end{Verbatim}
\end{tcolorbox}
        
    \begin{tcolorbox}[breakable, size=fbox, boxrule=1pt, pad at break*=1mm,colback=cellbackground, colframe=cellborder]
\prompt{In}{incolor}{6}{\boxspacing}
\begin{Verbatim}[commandchars=\\\{\}]
\PY{c+c1}{\PYZsh{}Visualisation de la consommation totale d\PYZsq{}électricité en énergie }
\PY{n}{figure}\PY{p}{(}\PY{n}{figsize}\PY{o}{=}\PY{p}{(}\PY{l+m+mi}{16}\PY{p}{,} \PY{l+m+mi}{6}\PY{p}{)}\PY{p}{)}
\PY{n}{plt}\PY{o}{.}\PY{n}{plot}\PY{p}{(}\PY{n}{consommation}\PY{o}{.}\PY{n}{consommationT}\PY{p}{)}
\PY{n}{plt}\PY{o}{.}\PY{n}{title}\PY{p}{(}\PY{l+s+s2}{\PYZdq{}}\PY{l+s+s2}{Consommation totale d}\PY{l+s+s2}{\PYZsq{}}\PY{l+s+s2}{électricité en France de 2010 à 2021}\PY{l+s+s2}{\PYZdq{}}\PY{p}{)}
\PY{n}{plt}\PY{o}{.}\PY{n}{xlabel}\PY{p}{(}\PY{l+s+s2}{\PYZdq{}}\PY{l+s+s2}{temps en mois}\PY{l+s+s2}{\PYZdq{}}\PY{p}{)} \PY{c+c1}{\PYZsh{}Titre de l\PYZsq{}axe des abscisses}
\PY{n}{plt}\PY{o}{.}\PY{n}{ylabel}\PY{p}{(}\PY{l+s+s2}{\PYZdq{}}\PY{l+s+s2}{Consommation d}\PY{l+s+s2}{\PYZsq{}}\PY{l+s+s2}{énergie en }\PY{l+s+s2}{\PYZdq{}}\PY{p}{)}\PY{c+c1}{\PYZsh{}Titre de l\PYZsq{}axe des ordonnées}
\PY{n}{plt}\PY{o}{.}\PY{n}{show}\PY{p}{(}\PY{p}{)}\PY{c+c1}{\PYZsh{}Tracer le graphe}
\end{Verbatim}
\end{tcolorbox}

    \begin{center}
    \adjustimage{max size={0.9\linewidth}{0.9\paperheight}}{output_11_0.png}
    \end{center}
    { \hspace*{\fill} \\}
    
    \begin{tcolorbox}[breakable, size=fbox, boxrule=1pt, pad at break*=1mm,colback=cellbackground, colframe=cellborder]
\prompt{In}{incolor}{7}{\boxspacing}
\begin{Verbatim}[commandchars=\\\{\}]
\PY{c+c1}{\PYZsh{}Consommation totale d\PYZsq{}électricité en France de 2010 à 2021 par année }
\PY{n}{figure}\PY{p}{(}\PY{n}{figsize}\PY{o}{=}\PY{p}{(}\PY{l+m+mi}{15}\PY{p}{,} \PY{l+m+mi}{5}\PY{p}{)}\PY{p}{)}
\PY{n}{consommation}\PY{p}{[}\PY{l+s+s1}{\PYZsq{}}\PY{l+s+s1}{consommationT}\PY{l+s+s1}{\PYZsq{}}\PY{p}{]}\PY{o}{.}\PY{n}{resample}\PY{p}{(}\PY{l+s+s1}{\PYZsq{}}\PY{l+s+s1}{Y}\PY{l+s+s1}{\PYZsq{}}\PY{p}{)}\PY{o}{.}\PY{n}{plot}\PY{p}{(}\PY{n}{title}\PY{o}{=}\PY{l+s+s1}{\PYZsq{}}\PY{l+s+s1}{Consommation totale d}\PY{l+s+se}{\PYZbs{}\PYZsq{}}\PY{l+s+s1}{électricité en France de 2010 à 2021 par année }\PY{l+s+s1}{\PYZsq{}}\PY{p}{)}
\PY{n}{plt}\PY{o}{.}\PY{n}{show}\PY{p}{(}\PY{p}{)}
\end{Verbatim}
\end{tcolorbox}

    \begin{center}
    \adjustimage{max size={0.9\linewidth}{0.9\paperheight}}{output_12_0.png}
    \end{center}
    { \hspace*{\fill} \\}
    
    \begin{tcolorbox}[breakable, size=fbox, boxrule=1pt, pad at break*=1mm,colback=cellbackground, colframe=cellborder]
\prompt{In}{incolor}{8}{\boxspacing}
\begin{Verbatim}[commandchars=\\\{\}]
\PY{c+c1}{\PYZsh{}La fonction aggregate pour rassembler les statistique(moyenne, écart type, min, max)}
\PY{n}{consommation}\PY{p}{[}\PY{l+s+s1}{\PYZsq{}}\PY{l+s+s1}{consommationT}\PY{l+s+s1}{\PYZsq{}}\PY{p}{]}\PY{o}{.}\PY{n}{resample}\PY{p}{(}\PY{l+s+s1}{\PYZsq{}}\PY{l+s+s1}{Y}\PY{l+s+s1}{\PYZsq{}}\PY{p}{)}\PY{o}{.}\PY{n}{agg}\PY{p}{(}\PY{p}{[}\PY{l+s+s1}{\PYZsq{}}\PY{l+s+s1}{mean}\PY{l+s+s1}{\PYZsq{}}\PY{p}{,} \PY{l+s+s1}{\PYZsq{}}\PY{l+s+s1}{std}\PY{l+s+s1}{\PYZsq{}}\PY{p}{,} \PY{l+s+s1}{\PYZsq{}}\PY{l+s+s1}{min}\PY{l+s+s1}{\PYZsq{}}\PY{p}{,} \PY{l+s+s1}{\PYZsq{}}\PY{l+s+s1}{max}\PY{l+s+s1}{\PYZsq{}}\PY{p}{]}\PY{p}{)}
\end{Verbatim}
\end{tcolorbox}

            \begin{tcolorbox}[breakable, size=fbox, boxrule=.5pt, pad at break*=1mm, opacityfill=0]
\prompt{Out}{outcolor}{8}{\boxspacing}
\begin{Verbatim}[commandchars=\\\{\}]
                    mean          std    min    max
mois
2010-12-31  42756.333333  8528.680611  33069  57600
2011-12-31  39898.916667  6996.660937  32625  53873
2012-12-31  40793.083333  7718.614801  32247  54476
2013-12-31  41236.833333  7937.459381  31591  53619
2014-12-31  38762.500000  6471.028154  31004  49359
2015-12-31  39670.833333  6866.459334  31603  52536
2016-12-31  40268.250000  7118.985948  32132  50670
2017-12-31  40167.333333  8086.311243  32110  57406
2018-12-31  39869.250000  7234.222923  32451  50202
2019-12-31  39478.250000  7070.561018  31505  54186
2020-12-31  40744.833333  7698.871533  31772  53403
2021-12-31  38401.727273  6889.040051  31184  52983
\end{Verbatim}
\end{tcolorbox}
        
    \begin{tcolorbox}[breakable, size=fbox, boxrule=1pt, pad at break*=1mm,colback=cellbackground, colframe=cellborder]
\prompt{In}{incolor}{9}{\boxspacing}
\begin{Verbatim}[commandchars=\\\{\}]
\PY{c+c1}{\PYZsh{}Récapitulatif des consommations moyennes annuelles}
\PY{n}{conso\PYZus{}annu}\PY{o}{=}\PY{n}{consommation}\PY{o}{.}\PY{n}{resample}\PY{p}{(}\PY{l+s+s1}{\PYZsq{}}\PY{l+s+s1}{Y}\PY{l+s+s1}{\PYZsq{}}\PY{p}{)}\PY{o}{.}\PY{n}{sum}\PY{p}{(}\PY{p}{)}
\PY{n+nb}{print}\PY{p}{(}\PY{n}{conso\PYZus{}annu}\PY{p}{)}
\PY{n}{figure}\PY{p}{(}\PY{n}{figsize}\PY{o}{=}\PY{p}{(}\PY{l+m+mi}{12}\PY{p}{,} \PY{l+m+mi}{6}\PY{p}{)}\PY{p}{)}
\PY{n}{plt}\PY{o}{.}\PY{n}{plot}\PY{p}{(}\PY{n}{conso\PYZus{}annu}\PY{o}{.}\PY{n}{consommationT}\PY{p}{)}
\PY{n}{plt}\PY{o}{.}\PY{n}{title}\PY{p}{(}\PY{l+s+s2}{\PYZdq{}}\PY{l+s+s2}{Récapitulatif des consommations moyennes annuelles}\PY{l+s+s2}{\PYZdq{}}\PY{p}{)}
\PY{n}{plt}\PY{o}{.}\PY{n}{xlim}\PY{p}{(}\PY{n}{pd}\PY{o}{.}\PY{n}{Timestamp}\PY{p}{(}\PY{l+s+s1}{\PYZsq{}}\PY{l+s+s1}{2010\PYZhy{}01\PYZhy{}01}\PY{l+s+s1}{\PYZsq{}}\PY{p}{)}\PY{p}{,} \PY{n}{pd}\PY{o}{.}\PY{n}{Timestamp}\PY{p}{(}\PY{l+s+s1}{\PYZsq{}}\PY{l+s+s1}{2022\PYZhy{}01\PYZhy{}01}\PY{l+s+s1}{\PYZsq{}}\PY{p}{)}\PY{p}{)}
\PY{n}{plt}\PY{o}{.}\PY{n}{show}\PY{p}{(}\PY{p}{)}\PY{c+c1}{\PYZsh{}Tracer le graphe}
\end{Verbatim}
\end{tcolorbox}

    \begin{Verbatim}[commandchars=\\\{\}]
            consommationT
mois
2010-12-31         513076
2011-12-31         478787
2012-12-31         489517
2013-12-31         494842
2014-12-31         465150
2015-12-31         476050
2016-12-31         483219
2017-12-31         482008
2018-12-31         478431
2019-12-31         473739
2020-12-31         488938
2021-12-31         422419
    \end{Verbatim}

    \begin{center}
    \adjustimage{max size={0.9\linewidth}{0.9\paperheight}}{output_14_1.png}
    \end{center}
    { \hspace*{\fill} \\}
    
    \hypertarget{collecte-et-pruxe9-traitement-de-donnuxe9es-de-production}{%
\section{Collecte et pré-traitement de données de
production}\label{collecte-et-pruxe9-traitement-de-donnuxe9es-de-production}}

Création du dataset de production mensuelle d'électricité en France

    \begin{tcolorbox}[breakable, size=fbox, boxrule=1pt, pad at break*=1mm,colback=cellbackground, colframe=cellborder]
\prompt{In}{incolor}{10}{\boxspacing}
\begin{Verbatim}[commandchars=\\\{\}]
\PY{c+c1}{\PYZsh{}On prend juste les données de la production d\PYZsq{}électricité la france}
\PY{c+c1}{\PYZsh{}Copier dans production}
\PY{n}{production}\PY{o}{=}\PY{n}{data\PYZus{}coso\PYZus{}prod}\PY{o}{.}\PY{n}{copy}\PY{p}{(}\PY{p}{)}
\PY{c+c1}{\PYZsh{}Trier et extraire seulement ce qui nous intéresse}
\PY{n}{production}\PY{o}{=}\PY{n}{production}\PY{p}{[}\PY{p}{[}\PY{l+s+s1}{\PYZsq{}}\PY{l+s+s1}{Mois}\PY{l+s+s1}{\PYZsq{}}\PY{p}{,} \PY{l+s+s1}{\PYZsq{}}\PY{l+s+s1}{Territoire}\PY{l+s+s1}{\PYZsq{}}\PY{p}{,} \PY{l+s+s1}{\PYZsq{}}\PY{l+s+s1}{Production totale}\PY{l+s+s1}{\PYZsq{}}\PY{p}{]}\PY{p}{]}
\PY{c+c1}{\PYZsh{}On récupère les information où Territoire=France et on laisse les données des régions}
\PY{n}{production}\PY{o}{=}\PY{n}{production}\PY{p}{[}\PY{n}{production}\PY{p}{[}\PY{l+s+s1}{\PYZsq{}}\PY{l+s+s1}{Territoire}\PY{l+s+s1}{\PYZsq{}}\PY{p}{]}\PY{o}{==}\PY{l+s+s1}{\PYZsq{}}\PY{l+s+s1}{France}\PY{l+s+s1}{\PYZsq{}}\PY{p}{]}
\PY{c+c1}{\PYZsh{}Eliminer les données manquantes}
\PY{n}{production}\PY{o}{.}\PY{n}{drop}\PY{p}{(}\PY{l+s+s1}{\PYZsq{}}\PY{l+s+s1}{Territoire}\PY{l+s+s1}{\PYZsq{}}\PY{p}{,} \PY{n}{axis}\PY{o}{=}\PY{l+m+mi}{1}\PY{p}{,} \PY{n}{inplace}\PY{o}{=}\PY{k+kc}{True}\PY{p}{)}
\PY{c+c1}{\PYZsh{}renommer les colonnes Mois et Production totale}
\PY{n}{production}\PY{o}{.}\PY{n}{rename}\PY{p}{(}\PY{n}{columns}\PY{o}{=}\PY{p}{\PYZob{}}\PY{l+s+s1}{\PYZsq{}}\PY{l+s+s1}{Mois}\PY{l+s+s1}{\PYZsq{}}\PY{p}{:} \PY{l+s+s1}{\PYZsq{}}\PY{l+s+s1}{mois}\PY{l+s+s1}{\PYZsq{}}\PY{p}{,} \PY{l+s+s1}{\PYZsq{}}\PY{l+s+s1}{Production totale}\PY{l+s+s1}{\PYZsq{}}\PY{p}{:} \PY{l+s+s1}{\PYZsq{}}\PY{l+s+s1}{productionT}\PY{l+s+s1}{\PYZsq{}}\PY{p}{\PYZcb{}}\PY{p}{,} \PY{n}{inplace}\PY{o}{=}\PY{k+kc}{True}\PY{p}{)}
\PY{c+c1}{\PYZsh{}Aperçu des données mensuelles de consommation totale d\PYZsq{}électricité de notre copy consommation}
\PY{n}{display}\PY{p}{(}\PY{n}{production}\PY{o}{.}\PY{n}{head}\PY{p}{(}\PY{p}{)}\PY{p}{)}
\PY{c+c1}{\PYZsh{}nobre de ligne et de colonne}
\PY{n}{display}\PY{p}{(}\PY{n}{production}\PY{o}{.}\PY{n}{shape}\PY{p}{)}
\end{Verbatim}
\end{tcolorbox}

    
    \begin{Verbatim}[commandchars=\\\{\}]
      mois  productionT
0  2010-01        56542
1  2010-02        50406
2  2010-03        51071
3  2010-04        41693
4  2010-05        39847
    \end{Verbatim}

    
    
    \begin{Verbatim}[commandchars=\\\{\}]
(143, 2)
    \end{Verbatim}

    
    \begin{tcolorbox}[breakable, size=fbox, boxrule=1pt, pad at break*=1mm,colback=cellbackground, colframe=cellborder]
\prompt{In}{incolor}{11}{\boxspacing}
\begin{Verbatim}[commandchars=\\\{\}]
\PY{c+c1}{\PYZsh{}Triez par les valeurs le long de chaque axe mois par ordre croissant}
\PY{n}{production}\PY{o}{.}\PY{n}{sort\PYZus{}values}\PY{p}{(}\PY{n}{by}\PY{o}{=}\PY{l+s+s1}{\PYZsq{}}\PY{l+s+s1}{mois}\PY{l+s+s1}{\PYZsq{}}\PY{p}{,} \PY{n}{ascending}\PY{o}{=}\PY{k+kc}{True}\PY{p}{)}\PY{o}{.}\PY{n}{head}\PY{p}{(}\PY{p}{)}
\end{Verbatim}
\end{tcolorbox}

            \begin{tcolorbox}[breakable, size=fbox, boxrule=.5pt, pad at break*=1mm, opacityfill=0]
\prompt{Out}{outcolor}{11}{\boxspacing}
\begin{Verbatim}[commandchars=\\\{\}]
      mois  productionT
0  2010-01        56542
1  2010-02        50406
2  2010-03        51071
3  2010-04        41693
4  2010-05        39847
\end{Verbatim}
\end{tcolorbox}
        
    \begin{tcolorbox}[breakable, size=fbox, boxrule=1pt, pad at break*=1mm,colback=cellbackground, colframe=cellborder]
\prompt{In}{incolor}{12}{\boxspacing}
\begin{Verbatim}[commandchars=\\\{\}]
\PY{c+c1}{\PYZsh{}Changement du type de Series en datetime}
\PY{c+c1}{\PYZsh{}Changement d\PYZsq{}index pour que la série temporelle soit reconnue}
\PY{n}{production}\PY{p}{[}\PY{l+s+s1}{\PYZsq{}}\PY{l+s+s1}{mois}\PY{l+s+s1}{\PYZsq{}}\PY{p}{]} \PY{o}{=} \PY{n}{pd}\PY{o}{.}\PY{n}{to\PYZus{}datetime}\PY{p}{(}\PY{n}{production}\PY{p}{[}\PY{l+s+s1}{\PYZsq{}}\PY{l+s+s1}{mois}\PY{l+s+s1}{\PYZsq{}}\PY{p}{]}\PY{p}{)}
\PY{n}{production}\PY{o}{.}\PY{n}{set\PYZus{}index}\PY{p}{(}\PY{l+s+s1}{\PYZsq{}}\PY{l+s+s1}{mois}\PY{l+s+s1}{\PYZsq{}}\PY{p}{,} \PY{n}{inplace}\PY{o}{=}\PY{k+kc}{True}\PY{p}{)}
\PY{c+c1}{\PYZsh{}Visualisation des données selon l\PYZsq{}indice temporel mois\PYZhy{}année}
\PY{c+c1}{\PYZsh{}print(consommation)}
\PY{n}{production}\PY{o}{.}\PY{n}{head}\PY{p}{(}\PY{p}{)}
\end{Verbatim}
\end{tcolorbox}

            \begin{tcolorbox}[breakable, size=fbox, boxrule=.5pt, pad at break*=1mm, opacityfill=0]
\prompt{Out}{outcolor}{12}{\boxspacing}
\begin{Verbatim}[commandchars=\\\{\}]
            productionT
mois
2010-01-01        56542
2010-02-01        50406
2010-03-01        51071
2010-04-01        41693
2010-05-01        39847
\end{Verbatim}
\end{tcolorbox}
        
    \begin{tcolorbox}[breakable, size=fbox, boxrule=1pt, pad at break*=1mm,colback=cellbackground, colframe=cellborder]
\prompt{In}{incolor}{13}{\boxspacing}
\begin{Verbatim}[commandchars=\\\{\}]
\PY{c+c1}{\PYZsh{}Visualisation de la consommation totale d\PYZsq{}électricité.}
\PY{n}{figure}\PY{p}{(}\PY{n}{figsize}\PY{o}{=}\PY{p}{(}\PY{l+m+mi}{15}\PY{p}{,} \PY{l+m+mi}{5}\PY{p}{)}\PY{p}{)}
\PY{n}{plt}\PY{o}{.}\PY{n}{plot}\PY{p}{(}\PY{n}{production}\PY{o}{.}\PY{n}{productionT}\PY{p}{,} \PY{n}{c}\PY{o}{=}\PY{l+s+s1}{\PYZsq{}}\PY{l+s+s1}{g}\PY{l+s+s1}{\PYZsq{}}\PY{p}{)}
\PY{n}{plt}\PY{o}{.}\PY{n}{title}\PY{p}{(}\PY{l+s+s2}{\PYZdq{}}\PY{l+s+s2}{Production totale d}\PY{l+s+s2}{\PYZsq{}}\PY{l+s+s2}{électricité en France de 2010 à 2021}\PY{l+s+s2}{\PYZdq{}}\PY{p}{)}
\PY{n}{plt}\PY{o}{.}\PY{n}{xlabel}\PY{p}{(}\PY{l+s+s2}{\PYZdq{}}\PY{l+s+s2}{années}\PY{l+s+s2}{\PYZdq{}}\PY{p}{)} \PY{c+c1}{\PYZsh{}Titre de l\PYZsq{}axe des abscisses}
\PY{n}{plt}\PY{o}{.}\PY{n}{ylabel}\PY{p}{(}\PY{l+s+s2}{\PYZdq{}}\PY{l+s+s2}{Production totale d}\PY{l+s+s2}{\PYZsq{}}\PY{l+s+s2}{énergie en France}\PY{l+s+s2}{\PYZdq{}}\PY{p}{)}\PY{c+c1}{\PYZsh{}Titre de l\PYZsq{}axe des ordonnées}
\PY{n}{plt}\PY{o}{.}\PY{n}{xlim}\PY{p}{(}\PY{n}{pd}\PY{o}{.}\PY{n}{Timestamp}\PY{p}{(}\PY{l+s+s1}{\PYZsq{}}\PY{l+s+s1}{2010\PYZhy{}01\PYZhy{}01}\PY{l+s+s1}{\PYZsq{}}\PY{p}{)}\PY{p}{,} \PY{n}{pd}\PY{o}{.}\PY{n}{Timestamp}\PY{p}{(}\PY{l+s+s1}{\PYZsq{}}\PY{l+s+s1}{2022\PYZhy{}01\PYZhy{}01}\PY{l+s+s1}{\PYZsq{}}\PY{p}{)}\PY{p}{)}
\PY{n}{plt}\PY{o}{.}\PY{n}{show}\PY{p}{(}\PY{p}{)}\PY{c+c1}{\PYZsh{}Tracer le graphe}
\end{Verbatim}
\end{tcolorbox}

    \begin{center}
    \adjustimage{max size={0.9\linewidth}{0.9\paperheight}}{output_19_0.png}
    \end{center}
    { \hspace*{\fill} \\}
    
    \begin{tcolorbox}[breakable, size=fbox, boxrule=1pt, pad at break*=1mm,colback=cellbackground, colframe=cellborder]
\prompt{In}{incolor}{14}{\boxspacing}
\begin{Verbatim}[commandchars=\\\{\}]
\PY{n}{figure}\PY{p}{(}\PY{n}{figsize}\PY{o}{=}\PY{p}{(}\PY{l+m+mi}{15}\PY{p}{,} \PY{l+m+mi}{5}\PY{p}{)}\PY{p}{)}
\PY{n}{plt}\PY{o}{.}\PY{n}{plot}\PY{p}{(}\PY{n}{consommation}\PY{o}{.}\PY{n}{consommationT}\PY{p}{,} \PY{n}{c}\PY{o}{=}\PY{l+s+s1}{\PYZsq{}}\PY{l+s+s1}{b}\PY{l+s+s1}{\PYZsq{}}\PY{p}{)}
\PY{n}{plt}\PY{o}{.}\PY{n}{title}\PY{p}{(}\PY{l+s+s1}{\PYZsq{}}\PY{l+s+s1}{Tendance de la consommation et de la production d}\PY{l+s+se}{\PYZbs{}\PYZsq{}}\PY{l+s+s1}{électricité en France}\PY{l+s+s1}{\PYZsq{}}\PY{p}{)}
\PY{n}{plt}\PY{o}{.}\PY{n}{plot}\PY{p}{(}\PY{n}{production}\PY{o}{.}\PY{n}{productionT}\PY{p}{,} \PY{n}{c}\PY{o}{=}\PY{l+s+s1}{\PYZsq{}}\PY{l+s+s1}{g}\PY{l+s+s1}{\PYZsq{}}\PY{p}{)}
\PY{n}{plt}\PY{o}{.}\PY{n}{grid}\PY{p}{(}\PY{k+kc}{True}\PY{p}{)}
\PY{n}{plt}\PY{o}{.}\PY{n}{show}\PY{p}{(}\PY{p}{)}\PY{c+c1}{\PYZsh{}Tracer le graphe}
\end{Verbatim}
\end{tcolorbox}

    \begin{center}
    \adjustimage{max size={0.9\linewidth}{0.9\paperheight}}{output_20_0.png}
    \end{center}
    { \hspace*{\fill} \\}
    
    \begin{tcolorbox}[breakable, size=fbox, boxrule=1pt, pad at break*=1mm,colback=cellbackground, colframe=cellborder]
\prompt{In}{incolor}{15}{\boxspacing}
\begin{Verbatim}[commandchars=\\\{\}]
\PY{c+c1}{\PYZsh{}Création d\PYZsq{}un échantillon de travail par jointure des deux précédents dataframe (Consommation, Production)}
\PY{c+c1}{\PYZsh{}inner = prendre les donnée de meme longueure comme clé \PYZdq{}mois\PYZdq{} avace la fonction merge}
\PY{n}{conso\PYZus{}prod} \PY{o}{=} \PY{n}{pd}\PY{o}{.}\PY{n}{merge}\PY{p}{(}\PY{n}{consommation}\PY{p}{,} \PY{n}{production}\PY{p}{,} \PY{n}{how}\PY{o}{=}\PY{l+s+s1}{\PYZsq{}}\PY{l+s+s1}{inner}\PY{l+s+s1}{\PYZsq{}}\PY{p}{,} \PY{n}{on}\PY{o}{=}\PY{l+s+s1}{\PYZsq{}}\PY{l+s+s1}{mois}\PY{l+s+s1}{\PYZsq{}}\PY{p}{)}
\PY{n}{conso\PYZus{}prod}\PY{o}{.}\PY{n}{head}\PY{p}{(}\PY{p}{)}
\end{Verbatim}
\end{tcolorbox}

            \begin{tcolorbox}[breakable, size=fbox, boxrule=.5pt, pad at break*=1mm, opacityfill=0]
\prompt{Out}{outcolor}{15}{\boxspacing}
\begin{Verbatim}[commandchars=\\\{\}]
            consommationT  productionT
mois
2010-01-01          56342        56542
2010-02-01          48698        50406
2010-03-01          48294        51071
2010-04-01          38637        41693
2010-05-01          37284        39847
\end{Verbatim}
\end{tcolorbox}
        
    \begin{tcolorbox}[breakable, size=fbox, boxrule=1pt, pad at break*=1mm,colback=cellbackground, colframe=cellborder]
\prompt{In}{incolor}{16}{\boxspacing}
\begin{Verbatim}[commandchars=\\\{\}]
\PY{c+c1}{\PYZsh{}Visualisation de la consommation et de la production}
\PY{n}{conso\PYZus{}prod}\PY{p}{[}\PY{p}{[}\PY{l+s+s1}{\PYZsq{}}\PY{l+s+s1}{consommationT}\PY{l+s+s1}{\PYZsq{}}\PY{p}{,} \PY{l+s+s1}{\PYZsq{}}\PY{l+s+s1}{productionT}\PY{l+s+s1}{\PYZsq{}}\PY{p}{]}\PY{p}{]}\PY{o}{.}\PY{n}{plot}\PY{p}{(}\PY{n}{subplots}\PY{o}{=}\PY{k+kc}{True}\PY{p}{,} \PY{n}{figsize}\PY{o}{=}\PY{p}{(}\PY{l+m+mi}{15}\PY{p}{,} \PY{l+m+mi}{5}\PY{p}{)}\PY{p}{)}
\PY{n}{plt}\PY{o}{.}\PY{n}{show}\PY{p}{(}\PY{p}{)}
\end{Verbatim}
\end{tcolorbox}

    \begin{center}
    \adjustimage{max size={0.9\linewidth}{0.9\paperheight}}{output_22_0.png}
    \end{center}
    { \hspace*{\fill} \\}
    
    En général, il y a toujours un équilibre entre l'offre et la demande
d'électricité en France, mais il peut y avoir des fluctuations au cours
du temps en fonction de différents facteurs, tels que la météo, les
besoins énergétiques saisonniers, limportation et l'exportation\ldots{}
La France produit environ 500 TWh d'électricité par an, tandis que sa
consommation est d'environ 550 TWh par an. La différence est comblée par
des importations d'électricité.

    C'est pour quoi, pour faire une prévision de la demande en électricité
on va utiliser les données de météo france sur l'utilisation du
chauffage en France.

Le degré jour unifié (DJU) est la différence entre la température
extérieure et une température de référence qui permet de réaliser des
estimations de consommations d'énergie thermique pour maintenir un
bâtiment confortable en proportion de la rigueur de l'hiver ou de la
chaleur de l'été.

Le degré jour est une valeur représentative de l'écart entre la
température d'une journée donnée et un seuil de température préétabli
(18 °C dans le cas des DJU ou Degré Jour Unifié). Sommés sur une
période, ils permettent de calculer les besoins de chauffage et de
climatisation d'un bâtiment.

    \hypertarget{cruxe9ation-du-dataset-dju-chauffage-mensuelle-en-france}{%
\section{Création du dataset dju chauffage mensuelle en
France}\label{cruxe9ation-du-dataset-dju-chauffage-mensuelle-en-france}}

    \begin{tcolorbox}[breakable, size=fbox, boxrule=1pt, pad at break*=1mm,colback=cellbackground, colframe=cellborder]
\prompt{In}{incolor}{17}{\boxspacing}
\begin{Verbatim}[commandchars=\\\{\}]
\PY{c+c1}{\PYZsh{}Voyons l\PYZsq{}impact du chauffage électrique}
\PY{c+c1}{\PYZsh{}Chargement du dataset \PYZsq{}dju.xlsx\PYZsq{}}
\PY{n}{data\PYZus{}chauff} \PY{o}{=} \PY{n}{pd}\PY{o}{.}\PY{n}{read\PYZus{}csv}\PY{p}{(}\PY{l+s+s1}{\PYZsq{}}\PY{l+s+s1}{chauff\PYZus{}calcul\PYZus{}DJU.xlsx \PYZhy{} DJU \PYZhy{} Mensuel.csv}\PY{l+s+s1}{\PYZsq{}}\PY{p}{,} \PY{n}{header}\PY{o}{=}\PY{k+kc}{None}\PY{p}{,} \PY{n}{skiprows}\PY{o}{=}\PY{l+m+mi}{11}\PY{p}{)}
\PY{c+c1}{\PYZsh{}A partir de la ligne 11 du file excel}
\PY{n}{data\PYZus{}chauff}\PY{o}{.}\PY{n}{head}\PY{p}{(}\PY{p}{)}
\end{Verbatim}
\end{tcolorbox}

            \begin{tcolorbox}[breakable, size=fbox, boxrule=.5pt, pad at break*=1mm, opacityfill=0]
\prompt{Out}{outcolor}{17}{\boxspacing}
\begin{Verbatim}[commandchars=\\\{\}]
       0      1      2      3      4      5     6    7    8     9      10  \textbackslash{}
0     NaN    JAN    FÉV    MAR    AVR    MAI   JUN  JUI  AOÛ   SEP    OCT
1  2022.0  385.4  269.8  226.7  171.9     41   1.7    0    0     0      0
2  2021.0  396.7  302.8    271  228.3  138.3  11.1  0.2  5.1    21  150.7
3  2020.0    339  249.6  268.6   81.4   65.7  20.6  0.9  4.5  34.3  157.5
4  2019.0  404.9  268.3  233.1  168.5  117.9  24.4    0  1.7  26.7  133.7

      11     12      13
0    NOV    DÉC   Total
1      0      0  1096.4
2  310.8  323.8  2159.4
3  227.2  336.8  1785.9
4  282.6  327.3    1989
\end{Verbatim}
\end{tcolorbox}
        
    \begin{tcolorbox}[breakable, size=fbox, boxrule=1pt, pad at break*=1mm,colback=cellbackground, colframe=cellborder]
\prompt{In}{incolor}{18}{\boxspacing}
\begin{Verbatim}[commandchars=\\\{\}]
\PY{c+c1}{\PYZsh{}Comme l\PYZsq{}année 2022 n\PYZsq{}est pas encore fini, on va devoir travailler jusqu\PYZsq{}à l\PYZsq{}année 2021 }
\PY{c+c1}{\PYZsh{}Suppression des deux premières lignes et dernière Series}
\PY{n}{data\PYZus{}chauff}\PY{o}{.}\PY{n}{drop}\PY{p}{(}\PY{p}{[}\PY{l+m+mi}{0}\PY{p}{,} \PY{l+m+mi}{1}\PY{p}{]}\PY{p}{,} \PY{n}{axis}\PY{o}{=}\PY{l+m+mi}{0}\PY{p}{,} \PY{n}{inplace}\PY{o}{=}\PY{k+kc}{True}\PY{p}{)}
\PY{n}{data\PYZus{}chauff}\PY{o}{.}\PY{n}{drop}\PY{p}{(}\PY{p}{[}\PY{l+m+mi}{13}\PY{p}{]}\PY{p}{,} \PY{n}{axis}\PY{o}{=}\PY{l+m+mi}{1}\PY{p}{,} \PY{n}{inplace}\PY{o}{=}\PY{k+kc}{True}\PY{p}{)}
\PY{n}{data\PYZus{}chauff}\PY{o}{.}\PY{n}{head}\PY{p}{(}\PY{l+m+mi}{10}\PY{p}{)}
\end{Verbatim}
\end{tcolorbox}

            \begin{tcolorbox}[breakable, size=fbox, boxrule=.5pt, pad at break*=1mm, opacityfill=0]
\prompt{Out}{outcolor}{18}{\boxspacing}
\begin{Verbatim}[commandchars=\\\{\}]
        0      1      2      3      4      5     6     7     8     9      10  \textbackslash{}
2   2021.0  396.7  302.8    271  228.3  138.3  11.1   0.2   5.1    21  150.7
3   2020.0    339  249.6  268.6   81.4   65.7  20.6   0.9   4.5  34.3  157.5
4   2019.0  404.9  268.3  233.1  168.5  117.9  24.4     0   1.7  26.7  133.7
5   2018.0  303.4  432.6  314.3  119.7   55.9   8.1     0   3.3  34.3  122.4
6   2017.0  467.9  278.4  206.1  182.6     75   9.4     1   6.8  62.6   99.4
7   2016.0  364.4  321.6  321.1  212.1   88.1  27.5   5.7   3.2  11.7    176
8   2015.0    392  365.7  275.5  141.1   91.5  15.8   6.9   6.1  71.9  176.9
9   2014.0  324.4  281.9  223.9  135.5  100.2  19.1   8.3  19.3    16   92.3
10  2013.0  429.2  402.2  376.6  209.5  158.4  43.6   0.6     5  41.5    105
11  2012.0    336  435.9  201.9  230.3   83.3    35  12.4   2.4    58  154.6

       11     12
2   310.8  323.8
3   227.2  336.8
4   282.6  327.3
5   282.5  325.9
6   282.6    369
7   285.6  390.8
8     195  248.1
9   222.6  368.2
10  303.9  349.5
11  296.2  345.9
\end{Verbatim}
\end{tcolorbox}
        
    \begin{tcolorbox}[breakable, size=fbox, boxrule=1pt, pad at break*=1mm,colback=cellbackground, colframe=cellborder]
\prompt{In}{incolor}{19}{\boxspacing}
\begin{Verbatim}[commandchars=\\\{\}]
\PY{c+c1}{\PYZsh{}La première series est considérée comme index}
\PY{n}{data\PYZus{}chauff}\PY{p}{[}\PY{l+m+mi}{0}\PY{p}{]} \PY{o}{=} \PY{n}{data\PYZus{}chauff}\PY{p}{[}\PY{l+m+mi}{0}\PY{p}{]}\PY{o}{.}\PY{n}{astype}\PY{p}{(}\PY{n+nb}{int}\PY{p}{)}
\PY{n}{data\PYZus{}chauff}\PY{o}{.}\PY{n}{set\PYZus{}index}\PY{p}{(}\PY{p}{[}\PY{l+m+mi}{0}\PY{p}{]}\PY{p}{,} \PY{n}{inplace}\PY{o}{=}\PY{k+kc}{True}\PY{p}{)}
\PY{n}{data\PYZus{}chauff}\PY{o}{.}\PY{n}{head}\PY{p}{(}\PY{l+m+mi}{10}\PY{p}{)}
\end{Verbatim}
\end{tcolorbox}

            \begin{tcolorbox}[breakable, size=fbox, boxrule=.5pt, pad at break*=1mm, opacityfill=0]
\prompt{Out}{outcolor}{19}{\boxspacing}
\begin{Verbatim}[commandchars=\\\{\}]
         1      2      3      4      5     6     7     8     9      10     11  \textbackslash{}
0
2021  396.7  302.8    271  228.3  138.3  11.1   0.2   5.1    21  150.7  310.8
2020    339  249.6  268.6   81.4   65.7  20.6   0.9   4.5  34.3  157.5  227.2
2019  404.9  268.3  233.1  168.5  117.9  24.4     0   1.7  26.7  133.7  282.6
2018  303.4  432.6  314.3  119.7   55.9   8.1     0   3.3  34.3  122.4  282.5
2017  467.9  278.4  206.1  182.6     75   9.4     1   6.8  62.6   99.4  282.6
2016  364.4  321.6  321.1  212.1   88.1  27.5   5.7   3.2  11.7    176  285.6
2015    392  365.7  275.5  141.1   91.5  15.8   6.9   6.1  71.9  176.9    195
2014  324.4  281.9  223.9  135.5  100.2  19.1   8.3  19.3    16   92.3  222.6
2013  429.2  402.2  376.6  209.5  158.4  43.6   0.6     5  41.5    105  303.9
2012    336  435.9  201.9  230.3   83.3    35  12.4   2.4    58  154.6  296.2

         12
0
2021  323.8
2020  336.8
2019  327.3
2018  325.9
2017    369
2016  390.8
2015  248.1
2014  368.2
2013  349.5
2012  345.9
\end{Verbatim}
\end{tcolorbox}
        
    \begin{tcolorbox}[breakable, size=fbox, boxrule=1pt, pad at break*=1mm,colback=cellbackground, colframe=cellborder]
\prompt{In}{incolor}{20}{\boxspacing}
\begin{Verbatim}[commandchars=\\\{\}]
\PY{c+c1}{\PYZsh{}Il est nécessaire de transformer ces données de manière à obtenir une matrice temporelle.}
\PY{c+c1}{\PYZsh{}Transformation des données }
\PY{n}{dju} \PY{o}{=} \PY{p}{\PYZob{}}\PY{l+s+s1}{\PYZsq{}}\PY{l+s+s1}{mois}\PY{l+s+s1}{\PYZsq{}}\PY{p}{:}\PY{p}{[}\PY{p}{]}\PY{p}{,}\PY{l+s+s1}{\PYZsq{}}\PY{l+s+s1}{dju\PYZus{}chauffage}\PY{l+s+s1}{\PYZsq{}}\PY{p}{:}\PY{p}{[}\PY{p}{]}\PY{p}{\PYZcb{}}
\PY{k}{for} \PY{n}{year} \PY{o+ow}{in} \PY{n}{data\PYZus{}chauff}\PY{o}{.}\PY{n}{index}\PY{o}{.}\PY{n}{values}\PY{p}{:}
    \PY{k}{for} \PY{n}{month} \PY{o+ow}{in} \PY{n}{data\PYZus{}chauff}\PY{o}{.}\PY{n}{columns}\PY{p}{:}
        \PY{n}{dju}\PY{p}{[}\PY{l+s+s1}{\PYZsq{}}\PY{l+s+s1}{mois}\PY{l+s+s1}{\PYZsq{}}\PY{p}{]}\PY{o}{.}\PY{n}{append}\PY{p}{(}\PY{l+s+sa}{f}\PY{l+s+s2}{\PYZdq{}}\PY{l+s+si}{\PYZob{}}\PY{n}{year}\PY{l+s+si}{\PYZcb{}}\PY{l+s+s2}{\PYZhy{}}\PY{l+s+si}{\PYZob{}}\PY{n}{month}\PY{l+s+si}{\PYZcb{}}\PY{l+s+s2}{\PYZdq{}}\PY{p}{)}
        \PY{n}{dju}\PY{p}{[}\PY{l+s+s1}{\PYZsq{}}\PY{l+s+s1}{dju\PYZus{}chauffage}\PY{l+s+s1}{\PYZsq{}}\PY{p}{]}\PY{o}{.}\PY{n}{append}\PY{p}{(}\PY{n}{data\PYZus{}chauff}\PY{o}{.}\PY{n}{loc}\PY{p}{[}\PY{n}{year}\PY{p}{,}\PY{n}{month}\PY{p}{]}\PY{p}{)}
\PY{n}{dju} \PY{o}{=} \PY{n}{pd}\PY{o}{.}\PY{n}{DataFrame}\PY{p}{(}\PY{n}{dju}\PY{p}{)}
\PY{n}{dju}\PY{o}{.}\PY{n}{head}\PY{p}{(}\PY{p}{)}
\end{Verbatim}
\end{tcolorbox}

            \begin{tcolorbox}[breakable, size=fbox, boxrule=.5pt, pad at break*=1mm, opacityfill=0]
\prompt{Out}{outcolor}{20}{\boxspacing}
\begin{Verbatim}[commandchars=\\\{\}]
     mois dju\_chauffage
0  2021-1         396.7
1  2021-2         302.8
2  2021-3           271
3  2021-4         228.3
4  2021-5         138.3
\end{Verbatim}
\end{tcolorbox}
        
    \begin{tcolorbox}[breakable, size=fbox, boxrule=1pt, pad at break*=1mm,colback=cellbackground, colframe=cellborder]
\prompt{In}{incolor}{21}{\boxspacing}
\begin{Verbatim}[commandchars=\\\{\}]
\PY{n}{dju}\PY{o}{.}\PY{n}{sort\PYZus{}values}\PY{p}{(}\PY{n}{by}\PY{o}{=}\PY{l+s+s1}{\PYZsq{}}\PY{l+s+s1}{mois}\PY{l+s+s1}{\PYZsq{}}\PY{p}{,} \PY{n}{ascending}\PY{o}{=}\PY{k+kc}{True}\PY{p}{)}\PY{o}{.}\PY{n}{head}\PY{p}{(}\PY{p}{)}
\PY{c+c1}{\PYZsh{}Indexation des données selon les mois d\PYZsq{}enregistrement}
\PY{n}{dju}\PY{p}{[}\PY{l+s+s1}{\PYZsq{}}\PY{l+s+s1}{mois}\PY{l+s+s1}{\PYZsq{}}\PY{p}{]} \PY{o}{=} \PY{n}{pd}\PY{o}{.}\PY{n}{to\PYZus{}datetime}\PY{p}{(}\PY{n}{dju}\PY{p}{[}\PY{l+s+s1}{\PYZsq{}}\PY{l+s+s1}{mois}\PY{l+s+s1}{\PYZsq{}}\PY{p}{]}\PY{p}{)}
\PY{n}{dju}\PY{o}{.}\PY{n}{set\PYZus{}index}\PY{p}{(}\PY{l+s+s1}{\PYZsq{}}\PY{l+s+s1}{mois}\PY{l+s+s1}{\PYZsq{}}\PY{p}{,} \PY{n}{inplace}\PY{o}{=}\PY{k+kc}{True}\PY{p}{)}
\PY{n+nb}{print}\PY{p}{(}\PY{n}{dju}\PY{p}{)}
\end{Verbatim}
\end{tcolorbox}

    \begin{Verbatim}[commandchars=\\\{\}]
           dju\_chauffage
mois
2021-01-01         396.7
2021-02-01         302.8
2021-03-01           271
2021-04-01         228.3
2021-05-01         138.3
{\ldots}                  {\ldots}
2010-08-01          11.1
2010-09-01          52.3
2010-10-01         172.2
2010-11-01           310
2010-12-01           512

[144 rows x 1 columns]
    \end{Verbatim}

    \begin{tcolorbox}[breakable, size=fbox, boxrule=1pt, pad at break*=1mm,colback=cellbackground, colframe=cellborder]
\prompt{In}{incolor}{22}{\boxspacing}
\begin{Verbatim}[commandchars=\\\{\}]
\PY{n}{dju\PYZus{}data}\PY{o}{=}\PY{n}{dju}\PY{o}{.}\PY{n}{sort\PYZus{}values}\PY{p}{(}\PY{n}{by}\PY{o}{=}\PY{l+s+s1}{\PYZsq{}}\PY{l+s+s1}{mois}\PY{l+s+s1}{\PYZsq{}}\PY{p}{,} \PY{n}{ascending}\PY{o}{=}\PY{k+kc}{True}\PY{p}{)}\PY{o}{.}\PY{n}{head}\PY{p}{(}\PY{l+m+mi}{144}\PY{p}{)}
\PY{n+nb}{print}\PY{p}{(}\PY{n}{dju\PYZus{}data}\PY{p}{)}
\end{Verbatim}
\end{tcolorbox}

    \begin{Verbatim}[commandchars=\\\{\}]
           dju\_chauffage
mois
2010-01-01         499.2
2010-02-01         371.4
2010-03-01         294.5
2010-04-01         165.3
2010-05-01         140.9
{\ldots}                  {\ldots}
2021-08-01           5.1
2021-09-01            21
2021-10-01         150.7
2021-11-01         310.8
2021-12-01         323.8

[144 rows x 1 columns]
    \end{Verbatim}

    \begin{tcolorbox}[breakable, size=fbox, boxrule=1pt, pad at break*=1mm,colback=cellbackground, colframe=cellborder]
\prompt{In}{incolor}{23}{\boxspacing}
\begin{Verbatim}[commandchars=\\\{\}]
\PY{c+c1}{\PYZsh{}L\PYZsq{}extraction des données traités}
\PY{c+c1}{\PYZsh{}Enregistrer les données dans un fichier csv}
\PY{n}{dju\PYZus{}data}\PY{o}{.}\PY{n}{to\PYZus{}csv}\PY{p}{(}\PY{l+s+s1}{\PYZsq{}}\PY{l+s+s1}{data\PYZus{}rec \PYZhy{} Feuille 1.csv}\PY{l+s+s1}{\PYZsq{}}\PY{p}{,} \PY{n}{index}\PY{o}{=}\PY{k+kc}{True}\PY{p}{,} \PY{n}{encoding}\PY{o}{=}\PY{l+s+s1}{\PYZsq{}}\PY{l+s+s1}{utf\PYZhy{}8}\PY{l+s+s1}{\PYZsq{}}\PY{p}{)}
\end{Verbatim}
\end{tcolorbox}

    \begin{tcolorbox}[breakable, size=fbox, boxrule=1pt, pad at break*=1mm,colback=cellbackground, colframe=cellborder]
\prompt{In}{incolor}{24}{\boxspacing}
\begin{Verbatim}[commandchars=\\\{\}]
\PY{c+c1}{\PYZsh{}Les donnée de l\PYZsq{}utilisation de chauffage en France en DJU=Degré jour unifié}

\PY{c+c1}{\PYZsh{}indexer par dates}
\PY{n}{dju\PYZus{}donnee}\PY{o}{=}\PY{n}{pd}\PY{o}{.}\PY{n}{read\PYZus{}csv}\PY{p}{(}\PY{l+s+s1}{\PYZsq{}}\PY{l+s+s1}{data\PYZus{}rec \PYZhy{} Feuille 1.csv}\PY{l+s+s1}{\PYZsq{}}\PY{p}{,} \PY{n}{index\PYZus{}col}\PY{o}{=}\PY{l+s+s1}{\PYZsq{}}\PY{l+s+s1}{mois}\PY{l+s+s1}{\PYZsq{}}\PY{p}{,} \PY{n}{parse\PYZus{}dates}\PY{o}{=}\PY{k+kc}{True}\PY{p}{)} 
\PY{n+nb}{print}\PY{p}{(}\PY{n}{dju\PYZus{}donnee}\PY{p}{)}
\end{Verbatim}
\end{tcolorbox}

    \begin{Verbatim}[commandchars=\\\{\}]
            dju\_chauffage
mois
2010-01-01          499.2
2010-02-01          371.4
2010-03-01          294.5
2010-04-01          165.3
2010-05-01          140.9
{\ldots}                   {\ldots}
2021-08-01            5.1
2021-09-01           21.0
2021-10-01          150.7
2021-11-01          310.8
2021-12-01          323.8

[144 rows x 1 columns]
    \end{Verbatim}

    \begin{tcolorbox}[breakable, size=fbox, boxrule=1pt, pad at break*=1mm,colback=cellbackground, colframe=cellborder]
\prompt{In}{incolor}{25}{\boxspacing}
\begin{Verbatim}[commandchars=\\\{\}]
\PY{c+c1}{\PYZsh{}Courbe sur l\PYZsq{}année 2019}
\PY{n}{dju\PYZus{}donnee}\PY{p}{[}\PY{l+s+s1}{\PYZsq{}}\PY{l+s+s1}{2019}\PY{l+s+s1}{\PYZsq{}}\PY{p}{]}\PY{o}{.}\PY{n}{plot}\PY{p}{(}\PY{n}{figsize}\PY{o}{=}\PY{p}{(}\PY{l+m+mi}{15}\PY{p}{,} \PY{l+m+mi}{5}\PY{p}{)}\PY{p}{,} \PY{n}{title}\PY{o}{=}\PY{l+s+s1}{\PYZsq{}}\PY{l+s+s1}{DJU chauffage sur une année}\PY{l+s+s1}{\PYZsq{}}\PY{p}{,} \PY{n}{color}\PY{o}{=}\PY{l+s+s1}{\PYZsq{}}\PY{l+s+s1}{r}\PY{l+s+s1}{\PYZsq{}}\PY{p}{)}
\PY{n}{plt}\PY{o}{.}\PY{n}{grid}\PY{p}{(}\PY{k+kc}{True}\PY{p}{)}
\PY{n}{plt}\PY{o}{.}\PY{n}{show}\PY{p}{(}\PY{p}{)}
\end{Verbatim}
\end{tcolorbox}

    \begin{center}
    \adjustimage{max size={0.9\linewidth}{0.9\paperheight}}{output_34_0.png}
    \end{center}
    { \hspace*{\fill} \\}
    
    \begin{tcolorbox}[breakable, size=fbox, boxrule=1pt, pad at break*=1mm,colback=cellbackground, colframe=cellborder]
\prompt{In}{incolor}{26}{\boxspacing}
\begin{Verbatim}[commandchars=\\\{\}]
\PY{c+c1}{\PYZsh{}Visualisation des données mensuelles des Degrés Jour Unifiés (DJU) chauffage électrique}
\PY{n}{figure}\PY{p}{(}\PY{n}{figsize}\PY{o}{=}\PY{p}{(}\PY{l+m+mi}{16}\PY{p}{,} \PY{l+m+mi}{6}\PY{p}{)}\PY{p}{)}
\PY{n}{dju\PYZus{}donnee}\PY{p}{[}\PY{l+s+s1}{\PYZsq{}}\PY{l+s+s1}{dju\PYZus{}chauffage}\PY{l+s+s1}{\PYZsq{}}\PY{p}{]}\PY{o}{.}\PY{n}{plot}\PY{p}{(}\PY{n}{title}\PY{o}{=}\PY{l+s+s1}{\PYZsq{}}\PY{l+s+s1}{DJU du chauffage de 2010 à 2021}\PY{l+s+s1}{\PYZsq{}}\PY{p}{,} \PY{n}{color}\PY{o}{=}\PY{l+s+s1}{\PYZsq{}}\PY{l+s+s1}{r}\PY{l+s+s1}{\PYZsq{}}\PY{p}{)}
\PY{n}{plt}\PY{o}{.}\PY{n}{show}\PY{p}{(}\PY{p}{)}
\end{Verbatim}
\end{tcolorbox}

    \begin{center}
    \adjustimage{max size={0.9\linewidth}{0.9\paperheight}}{output_35_0.png}
    \end{center}
    { \hspace*{\fill} \\}
    
    \begin{tcolorbox}[breakable, size=fbox, boxrule=1pt, pad at break*=1mm,colback=cellbackground, colframe=cellborder]
\prompt{In}{incolor}{27}{\boxspacing}
\begin{Verbatim}[commandchars=\\\{\}]
\PY{c+c1}{\PYZsh{}Afficher les donnée par année}
\PY{n}{plt}\PY{o}{.}\PY{n}{figure}\PY{p}{(}\PY{n}{figsize}\PY{o}{=}\PY{p}{(}\PY{l+m+mi}{16}\PY{p}{,} \PY{l+m+mi}{5}\PY{p}{)}\PY{p}{)}
\PY{n}{dju\PYZus{}donnee}\PY{p}{[}\PY{l+s+s1}{\PYZsq{}}\PY{l+s+s1}{dju\PYZus{}chauffage}\PY{l+s+s1}{\PYZsq{}}\PY{p}{]}\PY{o}{.}\PY{n}{resample}\PY{p}{(}\PY{l+s+s1}{\PYZsq{}}\PY{l+s+s1}{Y}\PY{l+s+s1}{\PYZsq{}}\PY{p}{)}\PY{o}{.}\PY{n}{plot}\PY{p}{(}\PY{n}{title}\PY{o}{=}\PY{l+s+s1}{\PYZsq{}}\PY{l+s+s1}{affichage par année}\PY{l+s+s1}{\PYZsq{}}\PY{p}{)}
\PY{n}{plt}\PY{o}{.}\PY{n}{show}\PY{p}{(}\PY{p}{)}
\end{Verbatim}
\end{tcolorbox}

    \begin{center}
    \adjustimage{max size={0.9\linewidth}{0.9\paperheight}}{output_36_0.png}
    \end{center}
    { \hspace*{\fill} \\}
    
    \begin{tcolorbox}[breakable, size=fbox, boxrule=1pt, pad at break*=1mm,colback=cellbackground, colframe=cellborder]
\prompt{In}{incolor}{28}{\boxspacing}
\begin{Verbatim}[commandchars=\\\{\}]
\PY{c+c1}{\PYZsh{}Courbe sur toute la durée de moyenne annuelle mean()= Moyenne}
\PY{n}{figure}\PY{p}{(}\PY{n}{figsize}\PY{o}{=}\PY{p}{(}\PY{l+m+mi}{12}\PY{p}{,} \PY{l+m+mi}{6}\PY{p}{)}\PY{p}{)}
\PY{n}{dju\PYZus{}donnee}\PY{p}{[}\PY{l+s+s1}{\PYZsq{}}\PY{l+s+s1}{dju\PYZus{}chauffage}\PY{l+s+s1}{\PYZsq{}}\PY{p}{]}\PY{o}{.}\PY{n}{resample}\PY{p}{(}\PY{l+s+s1}{\PYZsq{}}\PY{l+s+s1}{Y}\PY{l+s+s1}{\PYZsq{}}\PY{p}{)}\PY{o}{.}\PY{n}{mean}\PY{p}{(}\PY{p}{)}\PY{o}{.}\PY{n}{plot}\PY{p}{(}\PY{n}{title}\PY{o}{=}\PY{l+s+s1}{\PYZsq{}}\PY{l+s+s1}{Moyenne annuelle}\PY{l+s+s1}{\PYZsq{}}\PY{p}{,} \PY{n}{color}\PY{o}{=}\PY{l+s+s1}{\PYZsq{}}\PY{l+s+s1}{r}\PY{l+s+s1}{\PYZsq{}}\PY{p}{)}
\PY{n}{plt}\PY{o}{.}\PY{n}{show}\PY{p}{(}\PY{p}{)}
\end{Verbatim}
\end{tcolorbox}

    \begin{center}
    \adjustimage{max size={0.9\linewidth}{0.9\paperheight}}{output_37_0.png}
    \end{center}
    { \hspace*{\fill} \\}
    
    \hypertarget{un-peut-de-statistiques-sur-nos-donnuxe9es}{%
\subsection{Un peut de statistiques sur nos
données}\label{un-peut-de-statistiques-sur-nos-donnuxe9es}}

    \begin{tcolorbox}[breakable, size=fbox, boxrule=1pt, pad at break*=1mm,colback=cellbackground, colframe=cellborder]
\prompt{In}{incolor}{29}{\boxspacing}
\begin{Verbatim}[commandchars=\\\{\}]
\PY{c+c1}{\PYZsh{}On va afficher l\PYZsq{}écart type}
\PY{c+c1}{\PYZsh{}une mesure de la dispersion des valeurs d\PYZsq{}un échantillon statistique}
\PY{n}{figure}\PY{p}{(}\PY{n}{figsize}\PY{o}{=}\PY{p}{(}\PY{l+m+mi}{16}\PY{p}{,} \PY{l+m+mi}{5}\PY{p}{)}\PY{p}{)}
\PY{n}{dju\PYZus{}donnee}\PY{p}{[}\PY{l+s+s1}{\PYZsq{}}\PY{l+s+s1}{dju\PYZus{}chauffage}\PY{l+s+s1}{\PYZsq{}}\PY{p}{]}\PY{o}{.}\PY{n}{resample}\PY{p}{(}\PY{l+s+s1}{\PYZsq{}}\PY{l+s+s1}{Y}\PY{l+s+s1}{\PYZsq{}}\PY{p}{)}\PY{o}{.}\PY{n}{std}\PY{p}{(}\PY{p}{)}\PY{o}{.}\PY{n}{plot}\PY{p}{(}\PY{n}{color}\PY{o}{=}\PY{l+s+s1}{\PYZsq{}}\PY{l+s+s1}{r}\PY{l+s+s1}{\PYZsq{}}\PY{p}{)}
\PY{n}{plt}\PY{o}{.}\PY{n}{show}\PY{p}{(}\PY{p}{)}
\end{Verbatim}
\end{tcolorbox}

    \begin{center}
    \adjustimage{max size={0.9\linewidth}{0.9\paperheight}}{output_39_0.png}
    \end{center}
    { \hspace*{\fill} \\}
    
    \begin{tcolorbox}[breakable, size=fbox, boxrule=1pt, pad at break*=1mm,colback=cellbackground, colframe=cellborder]
\prompt{In}{incolor}{30}{\boxspacing}
\begin{Verbatim}[commandchars=\\\{\}]
\PY{n}{plt}\PY{o}{.}\PY{n}{figure}\PY{p}{(}\PY{n}{figsize}\PY{o}{=}\PY{p}{(}\PY{l+m+mi}{16}\PY{p}{,} \PY{l+m+mi}{8}\PY{p}{)}\PY{p}{)}
\PY{n}{dju\PYZus{}donnee}\PY{p}{[}\PY{l+s+s1}{\PYZsq{}}\PY{l+s+s1}{dju\PYZus{}chauffage}\PY{l+s+s1}{\PYZsq{}}\PY{p}{]}\PY{o}{.}\PY{n}{plot}\PY{p}{(}\PY{n}{color}\PY{o}{=}\PY{l+s+s1}{\PYZsq{}}\PY{l+s+s1}{r}\PY{l+s+s1}{\PYZsq{}}\PY{p}{)}
\PY{n}{dju\PYZus{}donnee}\PY{p}{[}\PY{l+s+s1}{\PYZsq{}}\PY{l+s+s1}{dju\PYZus{}chauffage}\PY{l+s+s1}{\PYZsq{}}\PY{p}{]}\PY{o}{.}\PY{n}{resample}\PY{p}{(}\PY{l+s+s1}{\PYZsq{}}\PY{l+s+s1}{Y}\PY{l+s+s1}{\PYZsq{}}\PY{p}{)}\PY{o}{.}\PY{n}{mean}\PY{p}{(}\PY{p}{)}\PY{o}{.}\PY{n}{plot}\PY{p}{(}\PY{p}{)}\PY{c+c1}{\PYZsh{}La moyenne}
\PY{n}{dju\PYZus{}donnee}\PY{p}{[}\PY{l+s+s1}{\PYZsq{}}\PY{l+s+s1}{dju\PYZus{}chauffage}\PY{l+s+s1}{\PYZsq{}}\PY{p}{]}\PY{o}{.}\PY{n}{resample}\PY{p}{(}\PY{l+s+s1}{\PYZsq{}}\PY{l+s+s1}{Y}\PY{l+s+s1}{\PYZsq{}}\PY{p}{)}\PY{o}{.}\PY{n}{std}\PY{p}{(}\PY{p}{)}\PY{o}{.}\PY{n}{plot}\PY{p}{(}\PY{p}{)}\PY{c+c1}{\PYZsh{}L\PYZsq{}ecart type}

\PY{c+c1}{\PYZsh{} Ajout de la légende}
\PY{n}{plt}\PY{o}{.}\PY{n}{legend}\PY{p}{(}\PY{p}{[}\PY{l+s+s1}{\PYZsq{}}\PY{l+s+s1}{Données}\PY{l+s+s1}{\PYZsq{}}\PY{p}{,} \PY{l+s+s1}{\PYZsq{}}\PY{l+s+s1}{Moyenne}\PY{l+s+s1}{\PYZsq{}}\PY{p}{,} \PY{l+s+s1}{\PYZsq{}}\PY{l+s+s1}{Écart type}\PY{l+s+s1}{\PYZsq{}}\PY{p}{]}\PY{p}{)}

\PY{n}{plt}\PY{o}{.}\PY{n}{show}\PY{p}{(}\PY{p}{)}
\end{Verbatim}
\end{tcolorbox}

    \begin{center}
    \adjustimage{max size={0.9\linewidth}{0.9\paperheight}}{output_40_0.png}
    \end{center}
    { \hspace*{\fill} \\}
    
    \begin{tcolorbox}[breakable, size=fbox, boxrule=1pt, pad at break*=1mm,colback=cellbackground, colframe=cellborder]
\prompt{In}{incolor}{31}{\boxspacing}
\begin{Verbatim}[commandchars=\\\{\}]
\PY{c+c1}{\PYZsh{}Statistiques descriptives de l\PYZsq{}échantillon dju}
\PY{n}{dju\PYZus{}donnee}\PY{o}{.}\PY{n}{describe}\PY{p}{(}\PY{p}{)}\PY{o}{.}\PY{n}{T}
\end{Verbatim}
\end{tcolorbox}

            \begin{tcolorbox}[breakable, size=fbox, boxrule=.5pt, pad at break*=1mm, opacityfill=0]
\prompt{Out}{outcolor}{31}{\boxspacing}
\begin{Verbatim}[commandchars=\\\{\}]
               count        mean         std  min     25\%     50\%     75\%  \textbackslash{}
dju\_chauffage  144.0  173.346528  143.640781  0.0  26.125  156.05  302.95

                 max
dju\_chauffage  512.0
\end{Verbatim}
\end{tcolorbox}
        
    \begin{tcolorbox}[breakable, size=fbox, boxrule=1pt, pad at break*=1mm,colback=cellbackground, colframe=cellborder]
\prompt{In}{incolor}{32}{\boxspacing}
\begin{Verbatim}[commandchars=\\\{\}]
\PY{c+c1}{\PYZsh{}Statistiques descriptives de l\PYZsq{}échantillon de la consommation d\PYZsq{}életricité en France}
\PY{n}{consommation}\PY{o}{.}\PY{n}{describe}\PY{p}{(}\PY{p}{)}\PY{o}{.}\PY{n}{T}
\end{Verbatim}
\end{tcolorbox}

            \begin{tcolorbox}[breakable, size=fbox, boxrule=.5pt, pad at break*=1mm, opacityfill=0]
\prompt{Out}{outcolor}{32}{\boxspacing}
\begin{Verbatim}[commandchars=\\\{\}]
               count          mean          std      min      25\%      50\%  \textbackslash{}
consommationT  143.0  40183.048951  7202.641671  31004.0  33858.5  37284.0

                   75\%      max
consommationT  45838.0  57600.0
\end{Verbatim}
\end{tcolorbox}
        
    \begin{tcolorbox}[breakable, size=fbox, boxrule=1pt, pad at break*=1mm,colback=cellbackground, colframe=cellborder]
\prompt{In}{incolor}{33}{\boxspacing}
\begin{Verbatim}[commandchars=\\\{\}]
\PY{c+c1}{\PYZsh{}La fonction aggregate pour rassemnbler les statistique(moyenne, ecart type, min, max)}
\PY{n}{dju\PYZus{}donnee}\PY{p}{[}\PY{l+s+s1}{\PYZsq{}}\PY{l+s+s1}{dju\PYZus{}chauffage}\PY{l+s+s1}{\PYZsq{}}\PY{p}{]}\PY{o}{.}\PY{n}{resample}\PY{p}{(}\PY{l+s+s1}{\PYZsq{}}\PY{l+s+s1}{Y}\PY{l+s+s1}{\PYZsq{}}\PY{p}{)}\PY{o}{.}\PY{n}{agg}\PY{p}{(}\PY{p}{[}\PY{l+s+s1}{\PYZsq{}}\PY{l+s+s1}{mean}\PY{l+s+s1}{\PYZsq{}}\PY{p}{,} \PY{l+s+s1}{\PYZsq{}}\PY{l+s+s1}{std}\PY{l+s+s1}{\PYZsq{}}\PY{p}{,} \PY{l+s+s1}{\PYZsq{}}\PY{l+s+s1}{min}\PY{l+s+s1}{\PYZsq{}}\PY{p}{,} \PY{l+s+s1}{\PYZsq{}}\PY{l+s+s1}{max}\PY{l+s+s1}{\PYZsq{}}\PY{p}{]}\PY{p}{)}
\end{Verbatim}
\end{tcolorbox}

            \begin{tcolorbox}[breakable, size=fbox, boxrule=.5pt, pad at break*=1mm, opacityfill=0]
\prompt{Out}{outcolor}{33}{\boxspacing}
\begin{Verbatim}[commandchars=\\\{\}]
                  mean         std   min    max
mois
2010-12-31  212.625000  183.352408   0.0  512.0
2011-12-31  150.775000  137.534849  11.9  392.0
2012-12-31  182.658333  147.496542   2.4  435.9
2013-12-31  202.083333  164.015741   0.6  429.2
2014-12-31  150.975000  129.480270   8.3  368.2
2015-12-31  165.541667  134.539609   6.1  392.0
2016-12-31  183.983333  151.379300   3.2  390.8
2017-12-31  170.066667  153.968228   1.0  467.9
2018-12-31  166.866667  154.787158   0.0  432.6
2019-12-31  165.758333  137.847791   0.0  404.9
2020-12-31  148.841667  130.015981   0.9  339.0
2021-12-31  179.983333  144.525347   0.2  396.7
\end{Verbatim}
\end{tcolorbox}
        
    \hypertarget{cruxe9ation-dun-uxe9chantillon-de-travail-par-jointure-des-deux-pruxe9cuxe9dents-dataframe}{%
\section{Création d'un échantillon de travail par jointure des deux
précédents
dataframe}\label{cruxe9ation-dun-uxe9chantillon-de-travail-par-jointure-des-deux-pruxe9cuxe9dents-dataframe}}

On va rassembler les données du dataset consommation et du dataset
dju\_donnee

    \begin{tcolorbox}[breakable, size=fbox, boxrule=1pt, pad at break*=1mm,colback=cellbackground, colframe=cellborder]
\prompt{In}{incolor}{34}{\boxspacing}
\begin{Verbatim}[commandchars=\\\{\}]
\PY{c+c1}{\PYZsh{}inner = prendre les donnée de meme longueure comme clé \PYZdq{}mois\PYZdq{} avace la fonction merge}
\PY{n}{conso\PYZus{}dju} \PY{o}{=} \PY{n}{pd}\PY{o}{.}\PY{n}{merge}\PY{p}{(}\PY{n}{consommation}\PY{p}{,} \PY{n}{dju\PYZus{}donnee}\PY{p}{,} \PY{n}{how}\PY{o}{=}\PY{l+s+s1}{\PYZsq{}}\PY{l+s+s1}{inner}\PY{l+s+s1}{\PYZsq{}}\PY{p}{,} \PY{n}{on}\PY{o}{=}\PY{l+s+s1}{\PYZsq{}}\PY{l+s+s1}{mois}\PY{l+s+s1}{\PYZsq{}}\PY{p}{)}
\PY{n+nb}{print}\PY{p}{(}\PY{n}{conso\PYZus{}dju}\PY{p}{)}
\end{Verbatim}
\end{tcolorbox}

    \begin{Verbatim}[commandchars=\\\{\}]
            consommationT  dju\_chauffage
mois
2010-01-01          56342          499.2
2010-02-01          48698          371.4
2010-03-01          48294          294.5
2010-04-01          38637          165.3
2010-05-01          37284          140.9
{\ldots}                   {\ldots}            {\ldots}
2021-07-01          32959            0.2
2021-08-01          31184            5.1
2021-09-01          32350           21.0
2021-10-01          36762          150.7
2021-11-01          44458          310.8

[143 rows x 2 columns]
    \end{Verbatim}

    \begin{tcolorbox}[breakable, size=fbox, boxrule=1pt, pad at break*=1mm,colback=cellbackground, colframe=cellborder]
\prompt{In}{incolor}{35}{\boxspacing}
\begin{Verbatim}[commandchars=\\\{\}]
\PY{c+c1}{\PYZsh{}Visualisation de la consommation et de l\PYZsq{}utilisation de chauffage}
\PY{n}{conso\PYZus{}dju}\PY{p}{[}\PY{p}{[}\PY{l+s+s1}{\PYZsq{}}\PY{l+s+s1}{consommationT}\PY{l+s+s1}{\PYZsq{}}\PY{p}{,} \PY{l+s+s1}{\PYZsq{}}\PY{l+s+s1}{dju\PYZus{}chauffage}\PY{l+s+s1}{\PYZsq{}}\PY{p}{]}\PY{p}{]}\PY{o}{.}\PY{n}{resample}\PY{p}{(}\PY{l+s+s1}{\PYZsq{}}\PY{l+s+s1}{Y}\PY{l+s+s1}{\PYZsq{}}\PY{p}{)}\PY{o}{.}\PY{n}{plot}\PY{p}{(}\PY{n}{figsize}\PY{o}{=}\PY{p}{(}\PY{l+m+mi}{16}\PY{p}{,} \PY{l+m+mi}{5}\PY{p}{)}\PY{p}{)}
\PY{n}{plt}\PY{o}{.}\PY{n}{title}\PY{p}{(}\PY{l+s+s1}{\PYZsq{}}\PY{l+s+s1}{Visualisation de la consommation et de l}\PY{l+s+se}{\PYZbs{}\PYZsq{}}\PY{l+s+s1}{utilisation de chauffage}\PY{l+s+s1}{\PYZsq{}}\PY{p}{)}
\PY{n}{plt}\PY{o}{.}\PY{n}{show}\PY{p}{(}\PY{p}{)}
\end{Verbatim}
\end{tcolorbox}

    \begin{center}
    \adjustimage{max size={0.9\linewidth}{0.9\paperheight}}{output_46_0.png}
    \end{center}
    { \hspace*{\fill} \\}
    
    Comme la lecture est compliqué, on va les afficher séparément en
respectant les échelles pour voir la tendance avec ``subplots=True''

    \begin{tcolorbox}[breakable, size=fbox, boxrule=1pt, pad at break*=1mm,colback=cellbackground, colframe=cellborder]
\prompt{In}{incolor}{36}{\boxspacing}
\begin{Verbatim}[commandchars=\\\{\}]
\PY{c+c1}{\PYZsh{}Visualisation de la consommation et de l\PYZsq{}utilisation de chauffage}
\PY{n}{conso\PYZus{}dju}\PY{p}{[}\PY{p}{[}\PY{l+s+s1}{\PYZsq{}}\PY{l+s+s1}{consommationT}\PY{l+s+s1}{\PYZsq{}}\PY{p}{,} \PY{l+s+s1}{\PYZsq{}}\PY{l+s+s1}{dju\PYZus{}chauffage}\PY{l+s+s1}{\PYZsq{}}\PY{p}{]}\PY{p}{]}\PY{o}{.}\PY{n}{plot}\PY{p}{(}\PY{n}{subplots}\PY{o}{=}\PY{k+kc}{True}\PY{p}{,} \PY{n}{figsize}\PY{o}{=}\PY{p}{(}\PY{l+m+mi}{16}\PY{p}{,} \PY{l+m+mi}{6}\PY{p}{)}\PY{p}{)}
\PY{n}{plt}\PY{o}{.}\PY{n}{show}\PY{p}{(}\PY{p}{)}
\end{Verbatim}
\end{tcolorbox}

    \begin{center}
    \adjustimage{max size={0.9\linewidth}{0.9\paperheight}}{output_48_0.png}
    \end{center}
    { \hspace*{\fill} \\}
    
    On voit que les deux courbes sont super bien corrélées. On va calculer
cette correction avec ``.corr()'' de pandas

    \begin{tcolorbox}[breakable, size=fbox, boxrule=1pt, pad at break*=1mm,colback=cellbackground, colframe=cellborder]
\prompt{In}{incolor}{37}{\boxspacing}
\begin{Verbatim}[commandchars=\\\{\}]
\PY{c+c1}{\PYZsh{}La matrice de correlation}
\PY{n}{conso\PYZus{}dju}\PY{p}{[}\PY{p}{[}\PY{l+s+s1}{\PYZsq{}}\PY{l+s+s1}{consommationT}\PY{l+s+s1}{\PYZsq{}}\PY{p}{,} \PY{l+s+s1}{\PYZsq{}}\PY{l+s+s1}{dju\PYZus{}chauffage}\PY{l+s+s1}{\PYZsq{}}\PY{p}{]}\PY{p}{]}\PY{o}{.}\PY{n}{corr}\PY{p}{(}\PY{p}{)}
\end{Verbatim}
\end{tcolorbox}

            \begin{tcolorbox}[breakable, size=fbox, boxrule=.5pt, pad at break*=1mm, opacityfill=0]
\prompt{Out}{outcolor}{37}{\boxspacing}
\begin{Verbatim}[commandchars=\\\{\}]
               consommationT  dju\_chauffage
consommationT       1.000000       0.970438
dju\_chauffage       0.970438       1.000000
\end{Verbatim}
\end{tcolorbox}
        
    Une corrélation de \textbf{97\%} signifie qu'il y a une forte relation
entre les deux variables que l'on étudie consommationT et
dju\_chauffage.

    \begin{tcolorbox}[breakable, size=fbox, boxrule=1pt, pad at break*=1mm,colback=cellbackground, colframe=cellborder]
\prompt{In}{incolor}{38}{\boxspacing}
\begin{Verbatim}[commandchars=\\\{\}]
\PY{c+c1}{\PYZsh{}visualiser notre dataset consommation et dju avec la librairie \PYZdq{}seaborn\PYZdq{}}
\PY{n}{figure}\PY{p}{(}\PY{n}{figsize}\PY{o}{=}\PY{p}{(}\PY{l+m+mi}{12}\PY{p}{,} \PY{l+m+mi}{6}\PY{p}{)}\PY{p}{)}
\PY{k+kn}{import} \PY{n+nn}{seaborn} \PY{k}{as} \PY{n+nn}{sns}
\PY{n}{sns}\PY{o}{.}\PY{n}{pairplot}\PY{p}{(}\PY{n}{conso\PYZus{}dju}\PY{p}{)}
\end{Verbatim}
\end{tcolorbox}

            \begin{tcolorbox}[breakable, size=fbox, boxrule=.5pt, pad at break*=1mm, opacityfill=0]
\prompt{Out}{outcolor}{38}{\boxspacing}
\begin{Verbatim}[commandchars=\\\{\}]
<seaborn.axisgrid.PairGrid at 0x7f25358f8bb0>
\end{Verbatim}
\end{tcolorbox}
        
    
    \begin{Verbatim}[commandchars=\\\{\}]
<Figure size 864x432 with 0 Axes>
    \end{Verbatim}

    
    \begin{center}
    \adjustimage{max size={0.9\linewidth}{0.9\paperheight}}{output_52_2.png}
    \end{center}
    { \hspace*{\fill} \\}
    
    Pour prédire la demande en électricité pour les années à venir, nous
allons utiliser un modèle de régression. Ce modèle nous permettra de
faire des prévisions précises sur la demande d'électricité.

    \begin{tcolorbox}[breakable, size=fbox, boxrule=1pt, pad at break*=1mm,colback=cellbackground, colframe=cellborder]
\prompt{In}{incolor}{39}{\boxspacing}
\begin{Verbatim}[commandchars=\\\{\}]
\PY{c+c1}{\PYZsh{}Préparation des données pour établir la Régression linéaire }
\PY{n}{y}\PY{o}{=}\PY{n}{conso\PYZus{}dju}\PY{p}{[}\PY{l+s+s1}{\PYZsq{}}\PY{l+s+s1}{consommationT}\PY{l+s+s1}{\PYZsq{}}\PY{p}{]}
\PY{n}{x}\PY{o}{=}\PY{n}{conso\PYZus{}dju}\PY{p}{[}\PY{l+s+s1}{\PYZsq{}}\PY{l+s+s1}{dju\PYZus{}chauffage}\PY{l+s+s1}{\PYZsq{}}\PY{p}{]}
\PY{n+nb}{print}\PY{p}{(}\PY{n}{x}\PY{p}{)}
\PY{n+nb}{print}\PY{p}{(}\PY{n}{y}\PY{p}{)}
\end{Verbatim}
\end{tcolorbox}

    \begin{Verbatim}[commandchars=\\\{\}]
mois
2010-01-01    499.2
2010-02-01    371.4
2010-03-01    294.5
2010-04-01    165.3
2010-05-01    140.9
              {\ldots}
2021-07-01      0.2
2021-08-01      5.1
2021-09-01     21.0
2021-10-01    150.7
2021-11-01    310.8
Name: dju\_chauffage, Length: 143, dtype: float64
mois
2010-01-01    56342
2010-02-01    48698
2010-03-01    48294
2010-04-01    38637
2010-05-01    37284
              {\ldots}
2021-07-01    32959
2021-08-01    31184
2021-09-01    32350
2021-10-01    36762
2021-11-01    44458
Name: consommationT, Length: 143, dtype: int64
    \end{Verbatim}

    \begin{tcolorbox}[breakable, size=fbox, boxrule=1pt, pad at break*=1mm,colback=cellbackground, colframe=cellborder]
\prompt{In}{incolor}{40}{\boxspacing}
\begin{Verbatim}[commandchars=\\\{\}]
\PY{n}{figure}\PY{p}{(}\PY{n}{figsize}\PY{o}{=}\PY{p}{(}\PY{l+m+mi}{15}\PY{p}{,} \PY{l+m+mi}{6}\PY{p}{)}\PY{p}{)}
\PY{n}{plt}\PY{o}{.}\PY{n}{scatter}\PY{p}{(}\PY{n}{x}\PY{p}{,} \PY{n}{y}\PY{p}{)}\PY{c+c1}{\PYZsh{}Permet d\PYZsq{}afficher le nuage de point afin de déterminer votre type de régression}
\PY{n}{plt}\PY{o}{.}\PY{n}{title}\PY{p}{(}\PY{l+s+s1}{\PYZsq{}}\PY{l+s+s1}{Évolution de de la consommation d}\PY{l+s+se}{\PYZbs{}\PYZsq{}}\PY{l+s+s1}{électricité en fonction du DJU chauffage}\PY{l+s+s1}{\PYZsq{}}\PY{p}{)}
\PY{n}{plt}\PY{o}{.}\PY{n}{xlabel}\PY{p}{(}\PY{l+s+s1}{\PYZsq{}}\PY{l+s+s1}{dju\PYZus{}chauffage}\PY{l+s+s1}{\PYZsq{}}\PY{p}{)}
\PY{n}{plt}\PY{o}{.}\PY{n}{ylabel}\PY{p}{(}\PY{l+s+s1}{\PYZsq{}}\PY{l+s+s1}{consommation d}\PY{l+s+se}{\PYZbs{}\PYZsq{}}\PY{l+s+s1}{électricité}\PY{l+s+s1}{\PYZsq{}}\PY{p}{)}
\PY{n}{plt}\PY{o}{.}\PY{n}{grid}\PY{p}{(}\PY{k+kc}{True}\PY{p}{)}
\PY{n}{plt}\PY{o}{.}\PY{n}{show}\PY{p}{(}\PY{p}{)}
\end{Verbatim}
\end{tcolorbox}

    \begin{center}
    \adjustimage{max size={0.9\linewidth}{0.9\paperheight}}{output_55_0.png}
    \end{center}
    { \hspace*{\fill} \\}
    
    On constate que nos points peuvent être reliés par une droite, donc une
\textbf{régression linéaire simple} paraît pertinente.

    \hypertarget{ruxe9gression-linuxe9aire-avec-sklearn}{%
\subsection{Régression Linéaire avec
SKLearn}\label{ruxe9gression-linuxe9aire-avec-sklearn}}

    \begin{tcolorbox}[breakable, size=fbox, boxrule=1pt, pad at break*=1mm,colback=cellbackground, colframe=cellborder]
\prompt{In}{incolor}{41}{\boxspacing}
\begin{Verbatim}[commandchars=\\\{\}]
\PY{k+kn}{from} \PY{n+nn}{sklearn}\PY{n+nn}{.}\PY{n+nn}{linear\PYZus{}model} \PY{k+kn}{import} \PY{n}{LinearRegression}
\PY{k+kn}{import} \PY{n+nn}{matplotlib}\PY{n+nn}{.}\PY{n+nn}{pyplot} \PY{k}{as} \PY{n+nn}{plt}
\end{Verbatim}
\end{tcolorbox}

    \begin{tcolorbox}[breakable, size=fbox, boxrule=1pt, pad at break*=1mm,colback=cellbackground, colframe=cellborder]
\prompt{In}{incolor}{42}{\boxspacing}
\begin{Verbatim}[commandchars=\\\{\}]
\PY{c+c1}{\PYZsh{}conso\PYZus{}dju}
\end{Verbatim}
\end{tcolorbox}

    \begin{tcolorbox}[breakable, size=fbox, boxrule=1pt, pad at break*=1mm,colback=cellbackground, colframe=cellborder]
\prompt{In}{incolor}{43}{\boxspacing}
\begin{Verbatim}[commandchars=\\\{\}]
\PY{n}{y}\PY{o}{=}\PY{n}{conso\PYZus{}dju}\PY{p}{[}\PY{l+s+s1}{\PYZsq{}}\PY{l+s+s1}{consommationT}\PY{l+s+s1}{\PYZsq{}}\PY{p}{]}
\PY{n}{X}\PY{o}{=}\PY{n}{conso\PYZus{}dju}\PY{o}{.}\PY{n}{drop}\PY{p}{(}\PY{p}{[}\PY{l+s+s1}{\PYZsq{}}\PY{l+s+s1}{consommationT}\PY{l+s+s1}{\PYZsq{}}\PY{p}{]}\PY{p}{,}\PY{n}{axis}\PY{o}{=} \PY{l+m+mi}{1}\PY{p}{)}\PY{c+c1}{\PYZsh{} X=dju\PYZus{}chauffage}
\PY{c+c1}{\PYZsh{}print(X, y)}
\end{Verbatim}
\end{tcolorbox}

    \begin{tcolorbox}[breakable, size=fbox, boxrule=1pt, pad at break*=1mm,colback=cellbackground, colframe=cellborder]
\prompt{In}{incolor}{44}{\boxspacing}
\begin{Verbatim}[commandchars=\\\{\}]
\PY{n}{figure}\PY{p}{(}\PY{n}{figsize}\PY{o}{=}\PY{p}{(}\PY{l+m+mi}{15}\PY{p}{,} \PY{l+m+mi}{6}\PY{p}{)}\PY{p}{)}
\PY{n}{plt}\PY{o}{.}\PY{n}{scatter}\PY{p}{(}\PY{n}{X}\PY{p}{,} \PY{n}{y}\PY{p}{)}
\PY{n}{plt}\PY{o}{.}\PY{n}{title}\PY{p}{(}\PY{l+s+s1}{\PYZsq{}}\PY{l+s+s1}{Évolution de de la consommation d}\PY{l+s+se}{\PYZbs{}\PYZsq{}}\PY{l+s+s1}{électricité en fonction du DJU chauffage}\PY{l+s+s1}{\PYZsq{}}\PY{p}{)}
\PY{n}{plt}\PY{o}{.}\PY{n}{xlabel}\PY{p}{(}\PY{l+s+s1}{\PYZsq{}}\PY{l+s+s1}{dju\PYZus{}chauffage}\PY{l+s+s1}{\PYZsq{}}\PY{p}{)}
\PY{n}{plt}\PY{o}{.}\PY{n}{ylabel}\PY{p}{(}\PY{l+s+s1}{\PYZsq{}}\PY{l+s+s1}{consommation d}\PY{l+s+se}{\PYZbs{}\PYZsq{}}\PY{l+s+s1}{électricité}\PY{l+s+s1}{\PYZsq{}}\PY{p}{)}
\PY{n}{plt}\PY{o}{.}\PY{n}{show}\PY{p}{(}\PY{p}{)}
\end{Verbatim}
\end{tcolorbox}

    \begin{center}
    \adjustimage{max size={0.9\linewidth}{0.9\paperheight}}{output_61_0.png}
    \end{center}
    { \hspace*{\fill} \\}
    
    \begin{tcolorbox}[breakable, size=fbox, boxrule=1pt, pad at break*=1mm,colback=cellbackground, colframe=cellborder]
\prompt{In}{incolor}{45}{\boxspacing}
\begin{Verbatim}[commandchars=\\\{\}]
\PY{n}{model}\PY{o}{=}\PY{n}{LinearRegression}\PY{p}{(}\PY{p}{)} \PY{c+c1}{\PYZsh{}Création du model de regression lineaire}
\PY{n}{model}\PY{o}{.}\PY{n}{fit}\PY{p}{(}\PY{n}{X}\PY{p}{,} \PY{n}{y}\PY{p}{)}\PY{c+c1}{\PYZsh{}Entrainer le modèle sur les données X, y divisées en deux tableaux}
\PY{n}{model}\PY{o}{.}\PY{n}{score}\PY{p}{(}\PY{n}{X}\PY{p}{,} \PY{n}{y}\PY{p}{)}\PY{c+c1}{\PYZsh{} Evaluer le modèle}
\end{Verbatim}
\end{tcolorbox}

            \begin{tcolorbox}[breakable, size=fbox, boxrule=.5pt, pad at break*=1mm, opacityfill=0]
\prompt{Out}{outcolor}{45}{\boxspacing}
\begin{Verbatim}[commandchars=\\\{\}]
0.9417490575514472
\end{Verbatim}
\end{tcolorbox}
        
    On obtient un score de 94\% c'est le coefficient de détermination est de
R²=1-(sum\_i=1\textsuperscript{n(yi-ŷi)²)/(sum\_i=1}n(yi-bar\{ŷ\}i)²) de
la méthode du moindre carrée.

    \begin{tcolorbox}[breakable, size=fbox, boxrule=1pt, pad at break*=1mm,colback=cellbackground, colframe=cellborder]
\prompt{In}{incolor}{46}{\boxspacing}
\begin{Verbatim}[commandchars=\\\{\}]
\PY{n}{prediction}\PY{o}{=}\PY{n}{model}\PY{o}{.}\PY{n}{predict}\PY{p}{(}\PY{n}{X}\PY{p}{)}\PY{c+c1}{\PYZsh{}Utiliser le model pour la prédire}
\PY{n}{figure}\PY{p}{(}\PY{n}{figsize}\PY{o}{=}\PY{p}{(}\PY{l+m+mi}{16}\PY{p}{,} \PY{l+m+mi}{6}\PY{p}{)}\PY{p}{)}
\PY{n}{plt}\PY{o}{.}\PY{n}{scatter}\PY{p}{(}\PY{n}{X}\PY{p}{,} \PY{n}{y}\PY{p}{)}\PY{c+c1}{\PYZsh{}Nuage de point }
\PY{n}{plt}\PY{o}{.}\PY{n}{plot}\PY{p}{(}\PY{n}{conso\PYZus{}dju}\PY{p}{[}\PY{l+s+s1}{\PYZsq{}}\PY{l+s+s1}{dju\PYZus{}chauffage}\PY{l+s+s1}{\PYZsq{}}\PY{p}{]}\PY{p}{,} \PY{n}{prediction}\PY{p}{,} \PY{n}{c}\PY{o}{=}\PY{l+s+s1}{\PYZsq{}}\PY{l+s+s1}{r}\PY{l+s+s1}{\PYZsq{}}\PY{p}{)}\PY{c+c1}{\PYZsh{}La droite de régression}
\PY{n}{plt}\PY{o}{.}\PY{n}{title}\PY{p}{(}\PY{l+s+s1}{\PYZsq{}}\PY{l+s+s1}{Prédiction de la consommation d}\PY{l+s+se}{\PYZbs{}\PYZsq{}}\PY{l+s+s1}{électricité en fonction des DJU}\PY{l+s+s1}{\PYZsq{}}\PY{p}{)}
\PY{n}{plt}\PY{o}{.}\PY{n}{xlabel}\PY{p}{(}\PY{l+s+s1}{\PYZsq{}}\PY{l+s+s1}{dju\PYZus{}chauffage}\PY{l+s+s1}{\PYZsq{}}\PY{p}{)}
\PY{n}{plt}\PY{o}{.}\PY{n}{ylabel}\PY{p}{(}\PY{l+s+s1}{\PYZsq{}}\PY{l+s+s1}{consommation d}\PY{l+s+se}{\PYZbs{}\PYZsq{}}\PY{l+s+s1}{électricité}\PY{l+s+s1}{\PYZsq{}}\PY{p}{)}
\PY{n}{plt}\PY{o}{.}\PY{n}{show}\PY{p}{(}\PY{p}{)}
\end{Verbatim}
\end{tcolorbox}

    \begin{center}
    \adjustimage{max size={0.9\linewidth}{0.9\paperheight}}{output_64_0.png}
    \end{center}
    { \hspace*{\fill} \\}
    
    \begin{tcolorbox}[breakable, size=fbox, boxrule=1pt, pad at break*=1mm,colback=cellbackground, colframe=cellborder]
\prompt{In}{incolor}{47}{\boxspacing}
\begin{Verbatim}[commandchars=\\\{\}]
\PY{n+nb}{print}\PY{p}{(}\PY{n}{prediction}\PY{p}{)}
\end{Verbatim}
\end{tcolorbox}

    \begin{Verbatim}[commandchars=\\\{\}]
[56096.49699525 49875.31664577 46131.89904895 39842.56805402
 38654.79966179 32896.07012077 31795.92398698 32336.26124738
 34341.83738508 40178.45337805 46886.42405221 56719.58861084
 50878.10471463 46633.29308337 43629.79678007 35573.41690654
 33908.59399611 33324.44560649 32526.10947401 32375.20447336
 32925.27754025 38007.36852996 42826.59274434 47017.85743987
 48152.07889639 53015.11423999 41624.22064237 43006.70516447
 35850.88739161 33499.69012338 32399.54398959 31912.75366491
 34619.30787015 39321.70240661 46214.65340415 48634.00131783
 52688.96472245 51374.6308458  50128.44761461 41994.18128913
 39506.68272999 33918.32980261 31825.13140646 32039.31914932
 33816.10383442 36907.22239617 46589.48195415 48809.24583471
 47587.40211976 45518.54323985 42695.15935667 38391.93288646
 36673.56304032 32725.69350713 32199.95995647 32735.42931362
 32574.78850648 36288.99868382 42631.87661446 49719.54374188
 50878.10471463 49597.8461607  45206.99743205 38664.53546828
 36250.05545785 32565.05269998 32131.80931101 32092.86608504
 35295.94642147 40407.24483066 41288.33531833 43873.19194241
 49534.5634185  47451.10082884 47426.76131261 42120.74677355
 36084.54674745 33134.59737986 32073.39447205 31951.69689088
 32365.46866686 40363.43370143 45698.65565998 50819.68987566
 54572.84327898 45348.16662621 41828.67257874 40684.71531573
 35446.85142212 32253.50689219 31844.60301945 32126.94140777
 34843.23141951 36634.61981435 45552.61856257 49758.48696785
 46565.14243792 52854.47343285 47095.74389182 37622.80417346
 34517.08190197 32190.22414998 31795.92398698 31956.56479413
 33465.61480065 37754.23756112 45547.75065933 47660.42066846
 51506.06423347 44856.50839827 43143.00645538 39998.34095792
 37535.18191502 32983.69237921 31795.92398698 31878.67834218
 33095.65415389 38304.31062802 45552.61856257 47728.57131391
 48298.1159938  43946.21049111 44871.11210802 35758.39722992
 34994.13642016 32798.71205583 31839.7351162  32014.97963309
 33465.61480065 39462.87160077 42855.80016382 48191.02212237
 51106.89616723 46535.93501844 44987.94178594 42909.34709954
 38528.23417737 32336.26124738 31805.65979348 32044.18705257
 32818.18366882 39131.85417998 46925.36727818]
    \end{Verbatim}

    \hypertarget{ruxe9gression-linuxe9aire-avec-statsmodels}{%
\section{Régression Linéaire avec
Statsmodels}\label{ruxe9gression-linuxe9aire-avec-statsmodels}}

    \begin{tcolorbox}[breakable, size=fbox, boxrule=1pt, pad at break*=1mm,colback=cellbackground, colframe=cellborder]
\prompt{In}{incolor}{48}{\boxspacing}
\begin{Verbatim}[commandchars=\\\{\}]
\PY{c+c1}{\PYZsh{}Préparation des données pour établir la Régression linéaire}
\PY{n}{y} \PY{o}{=} \PY{n}{conso\PYZus{}dju}\PY{p}{[}\PY{l+s+s1}{\PYZsq{}}\PY{l+s+s1}{consommationT}\PY{l+s+s1}{\PYZsq{}}\PY{p}{]}
\PY{n}{x} \PY{o}{=} \PY{n}{sm}\PY{o}{.}\PY{n}{add\PYZus{}constant}\PY{p}{(}\PY{n}{conso\PYZus{}dju}\PY{p}{[}\PY{l+s+s1}{\PYZsq{}}\PY{l+s+s1}{dju\PYZus{}chauffage}\PY{l+s+s1}{\PYZsq{}}\PY{p}{]}\PY{p}{)}
\end{Verbatim}
\end{tcolorbox}

    \begin{tcolorbox}[breakable, size=fbox, boxrule=1pt, pad at break*=1mm,colback=cellbackground, colframe=cellborder]
\prompt{In}{incolor}{49}{\boxspacing}
\begin{Verbatim}[commandchars=\\\{\}]
\PY{c+c1}{\PYZsh{}Régression linéaire Statsmodels}
\PY{c+c1}{\PYZsh{}Definition du modèle et entrainement}
\PY{n}{reg} \PY{o}{=} \PY{n}{sm}\PY{o}{.}\PY{n}{OLS}\PY{p}{(}\PY{n}{y}\PY{p}{,} \PY{n}{x}\PY{p}{)}\PY{o}{.}\PY{n}{fit}\PY{p}{(}\PY{p}{)}\PY{c+c1}{\PYZsh{}(Ordinary Least Squares)}
\PY{c+c1}{\PYZsh{}summary: affichage des caractéristique du modèle de régression construit}
\PY{n+nb}{print}\PY{p}{(}\PY{n}{reg}\PY{o}{.}\PY{n}{summary}\PY{p}{(}\PY{p}{)}\PY{p}{)}
\end{Verbatim}
\end{tcolorbox}

    \begin{Verbatim}[commandchars=\\\{\}]
                            OLS Regression Results
==============================================================================
Dep. Variable:          consommationT   R-squared:                       0.942
Model:                            OLS   Adj. R-squared:                  0.941
Method:                 Least Squares   F-statistic:                     2280.
Date:                Fri, 16 Dec 2022   Prob (F-statistic):           6.21e-89
Time:                        18:22:22   Log-Likelihood:                -1269.3
No. Observations:                 143   AIC:                             2543.
Df Residuals:                     141   BIC:                             2549.
Df Model:                           1
Covariance Type:            nonrobust
================================================================================
=
                    coef    std err          t      P>|t|      [0.025
0.975]
--------------------------------------------------------------------------------
-
const           3.18e+04    228.344    139.246      0.000    3.13e+04
3.22e+04
dju\_chauffage    48.6790      1.020     47.745      0.000      46.663
50.695
==============================================================================
Omnibus:                        3.141   Durbin-Watson:                   1.555
Prob(Omnibus):                  0.208   Jarque-Bera (JB):                2.828
Skew:                           0.201   Prob(JB):                        0.243
Kurtosis:                       3.560   Cond. No.                         351.
==============================================================================

Notes:
[1] Standard Errors assume that the covariance matrix of the errors is correctly
specified.
    \end{Verbatim}

    Cette sortie est un résumé des résultats d'une régression linéaire par
moindres carrés (OLS en anglais). On a modéliser la relation entre une
variable dépendante consommationT et une variables indépendantes
dju\_chauffage.

Le but de cette régression est de trouver les coefficients qui
minimisent l'\textbf{erreur} entre les valeurs prédites par le modèle et
les valeurs réelles de la variable dépendante.

\textbf{Le R-carré} (R-squared en anglais) indique la qualité du modèle.
Plus le R-carré est proche de 1, meilleure est la qualité du modèle.
Ici, le R-carré est de 0,942, ce qui suggère que le modèle explique bien
les variations de la variable dépendante.

\textbf{``coef''} donne les coefficients estimés pour chaque variable
indépendante. Ici, on peut voir que le coefficient associé à
dju\_chauffage est de 48,679. Cela signifie que pour chaque unité
supplémentaire de dju\_chauffage, on s'attend à ce que la consommationT
augmente de 48,679 unités.

\textbf{``std err''} donne l'erreur standard associée à chaque
coefficient. Plus l'erreur standard est faible, plus le coefficient est
fiable. Ici, l'erreur standard associée au coefficient de dju\_chauffage
est de 1,020, ce qui indique que le coefficient est assez fiable.

\textbf{``t''} donne la valeur du test t associé à chaque coefficient.
Plus la valeur du test t est élevée, plus le coefficient est
significatif. Ici, la valeur du test t associée au coefficient de
dju\_chauffage est de 47,745, ce qui suggère que le coefficient est
statistiquement significatif.

\textbf{``P\textgreater\textbar t\textbar{}''} donne la probabilité de
ne pas rejeter l'hypothèse nulle pour chaque coefficient. Plus cette
probabilité est faible, plus le coefficient est significatif. Ici, la
probabilité associée au coefficient de dju\_chauffage est de 6,21e-89,
ce qui indique que le coefficient est très significatif.

\textbf{``0.025 0.975''} donne les limites de confiance des coefficients
estimés. Cela signifie que nous pouvons être 95\% sûrs que la valeur
réelle du coefficient se trouve dans cette plage. Ici, nous pouvons être
95\% sûrs que la valeur réelle du coefficient de dju\_chauffage se
trouve entre 46,663 et 50

    \begin{tcolorbox}[breakable, size=fbox, boxrule=1pt, pad at break*=1mm,colback=cellbackground, colframe=cellborder]
\prompt{In}{incolor}{50}{\boxspacing}
\begin{Verbatim}[commandchars=\\\{\}]
\PY{n}{conso\PYZus{}dju}\PY{o}{.}\PY{n}{plot}\PY{p}{(}\PY{n}{x}\PY{o}{=}\PY{l+s+s1}{\PYZsq{}}\PY{l+s+s1}{dju\PYZus{}chauffage}\PY{l+s+s1}{\PYZsq{}}\PY{p}{,} \PY{n}{y}\PY{o}{=}\PY{l+s+s1}{\PYZsq{}}\PY{l+s+s1}{consommationT}\PY{l+s+s1}{\PYZsq{}}\PY{p}{,} \PY{n}{style}\PY{o}{=}\PY{l+s+s1}{\PYZsq{}}\PY{l+s+s1}{o}\PY{l+s+s1}{\PYZsq{}}\PY{p}{,} \PY{n}{figsize}\PY{o}{=}\PY{p}{(}\PY{l+m+mi}{15}\PY{p}{,} \PY{l+m+mi}{5}\PY{p}{)}\PY{p}{)}
\PY{n}{prediction}\PY{o}{=}\PY{n}{reg}\PY{o}{.}\PY{n}{predict}\PY{p}{(}\PY{n}{sm}\PY{o}{.}\PY{n}{add\PYZus{}constant}\PY{p}{(}\PY{n}{conso\PYZus{}dju}\PY{p}{[}\PY{l+s+s1}{\PYZsq{}}\PY{l+s+s1}{dju\PYZus{}chauffage}\PY{l+s+s1}{\PYZsq{}}\PY{p}{]}\PY{p}{)}\PY{p}{)}\PY{c+c1}{\PYZsh{}Pour la prédiction}
\PY{n}{plt}\PY{o}{.}\PY{n}{plot}\PY{p}{(}\PY{n}{conso\PYZus{}dju}\PY{p}{[}\PY{l+s+s1}{\PYZsq{}}\PY{l+s+s1}{dju\PYZus{}chauffage}\PY{l+s+s1}{\PYZsq{}}\PY{p}{]}\PY{p}{,} \PY{n}{prediction}\PY{p}{,} \PY{n}{label}\PY{o}{=}\PY{l+s+s2}{\PYZdq{}}\PY{l+s+s2}{Prédiction}\PY{l+s+s2}{\PYZdq{}}\PY{p}{)}
\PY{n}{plt}\PY{o}{.}\PY{n}{title}\PY{p}{(}\PY{l+s+s1}{\PYZsq{}}\PY{l+s+s1}{Prédiction de la consommation d}\PY{l+s+se}{\PYZbs{}\PYZsq{}}\PY{l+s+s1}{électricité en fonction des DJU de chauffage}\PY{l+s+s1}{\PYZsq{}}\PY{p}{)}
\PY{n}{plt}\PY{o}{.}\PY{n}{xlabel}\PY{p}{(}\PY{l+s+s1}{\PYZsq{}}\PY{l+s+s1}{dju\PYZus{}chauffage}\PY{l+s+s1}{\PYZsq{}}\PY{p}{)}
\PY{n}{plt}\PY{o}{.}\PY{n}{ylabel}\PY{p}{(}\PY{l+s+s1}{\PYZsq{}}\PY{l+s+s1}{consommation d}\PY{l+s+se}{\PYZbs{}\PYZsq{}}\PY{l+s+s1}{électricité}\PY{l+s+s1}{\PYZsq{}}\PY{p}{)}
\PY{n}{plt}\PY{o}{.}\PY{n}{grid}\PY{p}{(}\PY{p}{)}
\PY{n}{plt}\PY{o}{.}\PY{n}{legend}\PY{p}{(}\PY{p}{)}
\PY{n}{plt}\PY{o}{.}\PY{n}{show}\PY{p}{(}\PY{p}{)}
\end{Verbatim}
\end{tcolorbox}

    \begin{center}
    \adjustimage{max size={0.9\linewidth}{0.9\paperheight}}{output_70_0.png}
    \end{center}
    { \hspace*{\fill} \\}
    
    \begin{tcolorbox}[breakable, size=fbox, boxrule=1pt, pad at break*=1mm,colback=cellbackground, colframe=cellborder]
\prompt{In}{incolor}{51}{\boxspacing}
\begin{Verbatim}[commandchars=\\\{\}]
\PY{c+c1}{\PYZsh{}Coefficient de régression linéaire }
\PY{c+c1}{\PYZsh{}Affichage des paramètre de la droite de régression estimé dans y=ax+b}
\PY{n}{reg}\PY{o}{.}\PY{n}{params}\PY{p}{[}\PY{l+s+s1}{\PYZsq{}}\PY{l+s+s1}{dju\PYZus{}chauffage}\PY{l+s+s1}{\PYZsq{}}\PY{p}{]}\PY{c+c1}{\PYZsh{} affichage de a}
\end{Verbatim}
\end{tcolorbox}

            \begin{tcolorbox}[breakable, size=fbox, boxrule=.5pt, pad at break*=1mm, opacityfill=0]
\prompt{Out}{outcolor}{51}{\boxspacing}
\begin{Verbatim}[commandchars=\\\{\}]
48.67903246847878
\end{Verbatim}
\end{tcolorbox}
        
    \begin{tcolorbox}[breakable, size=fbox, boxrule=1pt, pad at break*=1mm,colback=cellbackground, colframe=cellborder]
\prompt{In}{incolor}{52}{\boxspacing}
\begin{Verbatim}[commandchars=\\\{\}]
\PY{c+c1}{\PYZsh{}Calcul de la correction : l\PYZsq{}effet DJU est retranché à la consommation totale}
\PY{c+c1}{\PYZsh{}Affichage de b}
\PY{c+c1}{\PYZsh{}b=y\PYZhy{}ax}
\PY{n}{conso\PYZus{}dju}\PY{p}{[}\PY{l+s+s1}{\PYZsq{}}\PY{l+s+s1}{conso\PYZus{}correction}\PY{l+s+s1}{\PYZsq{}}\PY{p}{]} \PY{o}{=} \PY{n}{conso\PYZus{}dju}\PY{p}{[}\PY{l+s+s1}{\PYZsq{}}\PY{l+s+s1}{consommationT}\PY{l+s+s1}{\PYZsq{}}\PY{p}{]} \PY{o}{\PYZhy{}} \PY{n}{conso\PYZus{}dju}\PY{p}{[}\PY{l+s+s1}{\PYZsq{}}\PY{l+s+s1}{dju\PYZus{}chauffage}\PY{l+s+s1}{\PYZsq{}}\PY{p}{]}\PY{o}{*}\PY{n}{reg}\PY{o}{.}\PY{n}{params}\PY{p}{[}\PY{l+s+s1}{\PYZsq{}}\PY{l+s+s1}{dju\PYZus{}chauffage}\PY{l+s+s1}{\PYZsq{}}\PY{p}{]}
\PY{n+nb}{print}\PY{p}{(}\PY{n}{conso\PYZus{}dju}\PY{p}{[}\PY{l+s+s1}{\PYZsq{}}\PY{l+s+s1}{conso\PYZus{}correction}\PY{l+s+s1}{\PYZsq{}}\PY{p}{]}\PY{p}{)}
\end{Verbatim}
\end{tcolorbox}

    \begin{Verbatim}[commandchars=\\\{\}]
mois
2010-01-01    32041.426992
2010-02-01    30618.607341
2010-03-01    33958.024938
2010-04-01    30590.355933
2010-05-01    30425.124325
                  {\ldots}
2021-07-01    32949.264194
2021-08-01    30935.736934
2021-09-01    31327.740318
2021-10-01    29426.069807
2021-11-01    29328.556709
Name: conso\_correction, Length: 143, dtype: float64
    \end{Verbatim}

    \begin{tcolorbox}[breakable, size=fbox, boxrule=1pt, pad at break*=1mm,colback=cellbackground, colframe=cellborder]
\prompt{In}{incolor}{53}{\boxspacing}
\begin{Verbatim}[commandchars=\\\{\}]
\PY{n}{figure}\PY{p}{(}\PY{n}{figsize}\PY{o}{=}\PY{p}{(}\PY{l+m+mi}{16}\PY{p}{,} \PY{l+m+mi}{5}\PY{p}{)}\PY{p}{,} \PY{n}{dpi}\PY{o}{=}\PY{l+m+mi}{80}\PY{p}{)}
\PY{c+c1}{\PYZsh{}Visualisation de la consommation en électricité avant et après correction}
\PY{n}{plt}\PY{o}{.}\PY{n}{plot}\PY{p}{(}\PY{n}{conso\PYZus{}dju}\PY{p}{[}\PY{l+s+s1}{\PYZsq{}}\PY{l+s+s1}{consommationT}\PY{l+s+s1}{\PYZsq{}}\PY{p}{]}\PY{p}{,} \PY{n}{label}\PY{o}{=}\PY{l+s+s1}{\PYZsq{}}\PY{l+s+s1}{ConsommationT}\PY{l+s+s1}{\PYZsq{}}\PY{p}{)}
\PY{n}{plt}\PY{o}{.}\PY{n}{plot}\PY{p}{(}\PY{n}{conso\PYZus{}dju}\PY{p}{[}\PY{l+s+s1}{\PYZsq{}}\PY{l+s+s1}{conso\PYZus{}correction}\PY{l+s+s1}{\PYZsq{}}\PY{p}{]}\PY{p}{,} \PY{n}{label}\PY{o}{=}\PY{l+s+s1}{\PYZsq{}}\PY{l+s+s1}{Consommation corrigée}\PY{l+s+s1}{\PYZsq{}}\PY{p}{)}
\PY{n}{plt}\PY{o}{.}\PY{n}{title}\PY{p}{(}\PY{l+s+s1}{\PYZsq{}}\PY{l+s+s1}{Consommation en électricité avant et après correction}\PY{l+s+s1}{\PYZsq{}}\PY{p}{)}
\PY{n}{plt}\PY{o}{.}\PY{n}{legend}\PY{p}{(}\PY{p}{)}
\PY{n}{plt}\PY{o}{.}\PY{n}{grid}\PY{p}{(}\PY{k+kc}{True}\PY{p}{)}
\PY{n}{plt}\PY{o}{.}\PY{n}{show}\PY{p}{(}\PY{p}{)}
\end{Verbatim}
\end{tcolorbox}

    \begin{center}
    \adjustimage{max size={0.9\linewidth}{0.9\paperheight}}{output_73_0.png}
    \end{center}
    { \hspace*{\fill} \\}
    
    \begin{tcolorbox}[breakable, size=fbox, boxrule=1pt, pad at break*=1mm,colback=cellbackground, colframe=cellborder]
\prompt{In}{incolor}{54}{\boxspacing}
\begin{Verbatim}[commandchars=\\\{\}]
\PY{c+c1}{\PYZsh{}Visualisation de la consommation et de l\PYZsq{}utilisation de chauffage}
\PY{n}{conso\PYZus{}dju}\PY{p}{[}\PY{p}{[}\PY{l+s+s1}{\PYZsq{}}\PY{l+s+s1}{consommationT}\PY{l+s+s1}{\PYZsq{}}\PY{p}{,} \PY{l+s+s1}{\PYZsq{}}\PY{l+s+s1}{conso\PYZus{}correction}\PY{l+s+s1}{\PYZsq{}}\PY{p}{]}\PY{p}{]}\PY{o}{.}\PY{n}{plot}\PY{p}{(}\PY{n}{subplots}\PY{o}{=}\PY{k+kc}{True}\PY{p}{,} \PY{n}{figsize}\PY{o}{=}\PY{p}{(}\PY{l+m+mi}{16}\PY{p}{,} \PY{l+m+mi}{5}\PY{p}{)}\PY{p}{)}
\PY{n}{plt}\PY{o}{.}\PY{n}{show}\PY{p}{(}\PY{p}{)}
\end{Verbatim}
\end{tcolorbox}

    \begin{center}
    \adjustimage{max size={0.9\linewidth}{0.9\paperheight}}{output_74_0.png}
    \end{center}
    { \hspace*{\fill} \\}
    
    \begin{tcolorbox}[breakable, size=fbox, boxrule=1pt, pad at break*=1mm,colback=cellbackground, colframe=cellborder]
\prompt{In}{incolor}{55}{\boxspacing}
\begin{Verbatim}[commandchars=\\\{\}]
\PY{c+c1}{\PYZsh{}Distribution des résidus}
\PY{n}{fig}\PY{p}{,} \PY{n}{ax} \PY{o}{=} \PY{n}{plt}\PY{o}{.}\PY{n}{subplots}\PY{p}{(}\PY{l+m+mi}{1}\PY{p}{,} \PY{l+m+mi}{2}\PY{p}{,} \PY{n}{figsize}\PY{o}{=}\PY{p}{(}\PY{l+m+mi}{15}\PY{p}{,}\PY{l+m+mi}{7}\PY{p}{)}\PY{p}{)}

\PY{n}{plt}\PY{o}{.}\PY{n}{hist}\PY{p}{(}\PY{n}{reg}\PY{o}{.}\PY{n}{resid}\PY{p}{,} \PY{n}{density}\PY{o}{=}\PY{k+kc}{True}\PY{p}{)}

\PY{n}{model\PYZus{}norm\PYZus{}residuals} \PY{o}{=} \PY{n}{reg}\PY{o}{.}\PY{n}{get\PYZus{}influence}\PY{p}{(}\PY{p}{)}\PY{o}{.}\PY{n}{resid\PYZus{}studentized\PYZus{}internal}
\PY{c+c1}{\PYZsh{}ŷ=âx+b\PYZca{} la droite de reg estimée}
\PY{n}{QQ} \PY{o}{=} \PY{n}{ProbPlot}\PY{p}{(}\PY{n}{model\PYZus{}norm\PYZus{}residuals}\PY{p}{)}
\PY{n}{QQ}\PY{o}{.}\PY{n}{qqplot}\PY{p}{(}\PY{n}{line}\PY{o}{=}\PY{l+s+s1}{\PYZsq{}}\PY{l+s+s1}{45}\PY{l+s+s1}{\PYZsq{}}\PY{p}{,} \PY{n}{alpha}\PY{o}{=}\PY{l+m+mf}{0.5}\PY{p}{,} \PY{n}{color}\PY{o}{=}\PY{l+s+s1}{\PYZsq{}}\PY{l+s+s1}{\PYZsh{}4C72B0}\PY{l+s+s1}{\PYZsq{}}\PY{p}{,} \PY{n}{ax}\PY{o}{=}\PY{n}{ax}\PY{p}{[}\PY{l+m+mi}{0}\PY{p}{]}\PY{p}{)}

\PY{n}{ax}\PY{p}{[}\PY{l+m+mi}{0}\PY{p}{]}\PY{o}{.}\PY{n}{set\PYZus{}title}\PY{p}{(}\PY{l+s+s1}{\PYZsq{}}\PY{l+s+s1}{Q\PYZhy{}Q Plot}\PY{l+s+s1}{\PYZsq{}}\PY{p}{)}
\PY{n}{ax}\PY{p}{[}\PY{l+m+mi}{1}\PY{p}{]}\PY{o}{.}\PY{n}{set\PYZus{}title}\PY{p}{(}\PY{l+s+s1}{\PYZsq{}}\PY{l+s+s1}{Histogramme des résidus}\PY{l+s+s1}{\PYZsq{}}\PY{p}{)}
\PY{n}{ax}\PY{p}{[}\PY{l+m+mi}{1}\PY{p}{]}\PY{o}{.}\PY{n}{set\PYZus{}xlabel}\PY{p}{(}\PY{l+s+s1}{\PYZsq{}}\PY{l+s+s1}{Valeurs résiduelles}\PY{l+s+s1}{\PYZsq{}}\PY{p}{)}
\PY{n}{ax}\PY{p}{[}\PY{l+m+mi}{1}\PY{p}{]}\PY{o}{.}\PY{n}{set\PYZus{}ylabel}\PY{p}{(}\PY{l+s+s1}{\PYZsq{}}\PY{l+s+s1}{Nombre de résidus}\PY{l+s+s1}{\PYZsq{}}\PY{p}{)}
                
\PY{n}{plt}\PY{o}{.}\PY{n}{show}\PY{p}{(}\PY{p}{)}
\end{Verbatim}
\end{tcolorbox}

    \begin{center}
    \adjustimage{max size={0.9\linewidth}{0.9\paperheight}}{output_75_0.png}
    \end{center}
    { \hspace*{\fill} \\}
    
    En régression linéaire, on cherche à modéliser la relation entre une
variable indépendante et une variable dépendante à l'aide d'une droite.
Le résidu d'une observation est la différence entre la valeur observée
de la variable dépendante et la valeur prédite par la droite de
régression pour la valeur de la variable indépendante correspondante. La
normalité des résidus est un critère important pour vérifier la validité
de la régression linéaire.

Si les résidus ne sont pas distribués de manière normale, cela peut
signifier que la régression linéaire n'est pas appropriée pour modéliser
la relation entre les variables

    \begin{tcolorbox}[breakable, size=fbox, boxrule=1pt, pad at break*=1mm,colback=cellbackground, colframe=cellborder]
\prompt{In}{incolor}{56}{\boxspacing}
\begin{Verbatim}[commandchars=\\\{\}]
\PY{c+c1}{\PYZsh{}Verification de l\PYZsq{}hypothèse de normalité de residu Epsilon= min(y\PYZhy{}(ax+b))}
\PY{n}{figure}\PY{p}{(}\PY{n}{figsize}\PY{o}{=}\PY{p}{(}\PY{l+m+mi}{10}\PY{p}{,} \PY{l+m+mi}{5}\PY{p}{)}\PY{p}{)}
\PY{n}{sns}\PY{o}{.}\PY{n}{distplot}\PY{p}{(}\PY{n}{reg}\PY{o}{.}\PY{n}{resid}\PY{p}{,} \PY{n}{color}\PY{o}{=}\PY{l+s+s1}{\PYZsq{}}\PY{l+s+s1}{red}\PY{l+s+s1}{\PYZsq{}}\PY{p}{)}
\PY{n}{plt}\PY{o}{.}\PY{n}{show}\PY{p}{(}\PY{p}{)}
\end{Verbatim}
\end{tcolorbox}

    \begin{center}
    \adjustimage{max size={0.9\linewidth}{0.9\paperheight}}{output_77_0.png}
    \end{center}
    { \hspace*{\fill} \\}
    
    La distribution des résidus est satisfaisante. En effet les résidus
suivent bien la distribution théorique d'une loi normale

    \hypertarget{suxe9rie-temporelle-avec-le-moduxe8le-additif}{%
\section{Série temporelle avec le modèle
additif}\label{suxe9rie-temporelle-avec-le-moduxe8le-additif}}

    Un modèle de série temporelle est une équation qui précise comment les
composantes de la tendance, de la saisonnalité et du bruit s'intègrent
pour créer cette série chronologique.

Par exemple X(t)=Tendance(t)+S(t)+epsilon(t) epsilon(t):le bruit s(t):
variation saisonnière selon la période t Tendance: l'évolution générale
dans le temps

    \begin{tcolorbox}[breakable, size=fbox, boxrule=1pt, pad at break*=1mm,colback=cellbackground, colframe=cellborder]
\prompt{In}{incolor}{57}{\boxspacing}
\begin{Verbatim}[commandchars=\\\{\}]
\PY{k+kn}{from} \PY{n+nn}{statsmodels}\PY{n+nn}{.}\PY{n+nn}{tsa}\PY{n+nn}{.}\PY{n+nn}{seasonal} \PY{k+kn}{import} \PY{n}{seasonal\PYZus{}decompose}
\PY{k+kn}{import} \PY{n+nn}{matplotlib}\PY{n+nn}{.}\PY{n+nn}{pyplot} \PY{k}{as} \PY{n+nn}{plt}
\PY{c+c1}{\PYZsh{}On calcul l\PYZsq{}ecart saisonnier entre deux pic ou deux creu}
\PY{n}{decomposition} \PY{o}{=} \PY{n}{seasonal\PYZus{}decompose}\PY{p}{(}\PY{n}{conso\PYZus{}dju}\PY{p}{[}\PY{l+s+s1}{\PYZsq{}}\PY{l+s+s1}{conso\PYZus{}correction}\PY{l+s+s1}{\PYZsq{}}\PY{p}{]}\PY{p}{,}  \PY{n}{model}\PY{o}{=}\PY{l+s+s1}{\PYZsq{}}\PY{l+s+s1}{additive}\PY{l+s+s1}{\PYZsq{}}\PY{p}{)}

\PY{n}{fig}\PY{p}{,} \PY{n}{ax} \PY{o}{=} \PY{n}{plt}\PY{o}{.}\PY{n}{subplots}\PY{p}{(}\PY{l+m+mi}{4}\PY{p}{,}\PY{l+m+mi}{1}\PY{p}{,} \PY{n}{figsize}\PY{o}{=}\PY{p}{(}\PY{l+m+mi}{16}\PY{p}{,}\PY{l+m+mi}{6}\PY{p}{)}\PY{p}{)}
\PY{n}{ax}\PY{p}{[}\PY{l+m+mi}{0}\PY{p}{]}\PY{o}{.}\PY{n}{plot}\PY{p}{(}\PY{n}{decomposition}\PY{o}{.}\PY{n}{observed}\PY{p}{)}
\PY{n}{ax}\PY{p}{[}\PY{l+m+mi}{0}\PY{p}{]}\PY{o}{.}\PY{n}{set\PYZus{}title}\PY{p}{(}\PY{l+s+s1}{\PYZsq{}}\PY{l+s+s1}{Série}\PY{l+s+s1}{\PYZsq{}}\PY{p}{,} \PY{n}{fontsize}\PY{o}{=}\PY{l+m+mi}{15}\PY{p}{)}\PY{c+c1}{\PYZsh{} serie temporelle X(t)}
\PY{n}{ax}\PY{p}{[}\PY{l+m+mi}{1}\PY{p}{]}\PY{o}{.}\PY{n}{plot}\PY{p}{(}\PY{n}{decomposition}\PY{o}{.}\PY{n}{trend}\PY{p}{)}
\PY{n}{ax}\PY{p}{[}\PY{l+m+mi}{1}\PY{p}{]}\PY{o}{.}\PY{n}{set\PYZus{}title}\PY{p}{(}\PY{l+s+s1}{\PYZsq{}}\PY{l+s+s1}{Tendance}\PY{l+s+s1}{\PYZsq{}}\PY{p}{,} \PY{n}{fontsize}\PY{o}{=}\PY{l+m+mi}{15}\PY{p}{)}\PY{c+c1}{\PYZsh{}Tandance: levolution générale dans le temps}
\PY{n}{ax}\PY{p}{[}\PY{l+m+mi}{2}\PY{p}{]}\PY{o}{.}\PY{n}{plot}\PY{p}{(}\PY{n}{decomposition}\PY{o}{.}\PY{n}{seasonal}\PY{p}{)}
\PY{n}{ax}\PY{p}{[}\PY{l+m+mi}{2}\PY{p}{]}\PY{o}{.}\PY{n}{set\PYZus{}title}\PY{p}{(}\PY{l+s+s1}{\PYZsq{}}\PY{l+s+s1}{Saisonnalité}\PY{l+s+s1}{\PYZsq{}}\PY{p}{,} \PY{n}{fontsize}\PY{o}{=}\PY{l+m+mi}{15}\PY{p}{)}\PY{c+c1}{\PYZsh{}Saisonnalité  S(t)=Y\PYZhy{}tandance}
\PY{n}{ax}\PY{p}{[}\PY{l+m+mi}{3}\PY{p}{]}\PY{o}{.}\PY{n}{plot}\PY{p}{(}\PY{n}{decomposition}\PY{o}{.}\PY{n}{resid}\PY{p}{,}\PY{l+s+s1}{\PYZsq{}}\PY{l+s+s1}{bo}\PY{l+s+s1}{\PYZsq{}}\PY{p}{)}
\PY{n}{ax}\PY{p}{[}\PY{l+m+mi}{3}\PY{p}{]}\PY{o}{.}\PY{n}{set\PYZus{}title}\PY{p}{(}\PY{l+s+s1}{\PYZsq{}}\PY{l+s+s1}{Résidus}\PY{l+s+s1}{\PYZsq{}}\PY{p}{,} \PY{n}{fontsize}\PY{o}{=}\PY{l+m+mi}{15}\PY{p}{)}\PY{c+c1}{\PYZsh{}Residu= X(t)\PYZhy{}Tendance(t)\PYZhy{}S(t)}
\PY{n}{plt}\PY{o}{.}\PY{n}{show}\PY{p}{(}\PY{p}{)}
\end{Verbatim}
\end{tcolorbox}

    \begin{center}
    \adjustimage{max size={0.9\linewidth}{0.9\paperheight}}{output_81_0.png}
    \end{center}
    { \hspace*{\fill} \\}
    
    Comme on a définit, la série temporelle, nous pouvons prédire la
consommation d'énergie pour les prochaines années en prenant en compte
l'utilisation du chauffage électrique.

    \hypertarget{pruxe9vision-par-lissage-exponentiel-par-la-muxe9thode-de-holt-winters}{%
\subsection{Prévision par lissage exponentiel par la méthode de
Holt-Winters}\label{pruxe9vision-par-lissage-exponentiel-par-la-muxe9thode-de-holt-winters}}

Le modèle additif de Holt-Winters est une extension du lissage
exponentiel de Holt qui permet de capturer la saisonnalité. Cette
méthode utilise des valeurs lissées exponentiellement pour prévoir le
niveau, la tendance et l'ajustement saisonnier. Cette approche additive
saisonnière ajoute le facteur de saisonnalité à la prévision à
tendances, ce qui donne lieu à la prévision additive de Holt-Winters.

    \begin{tcolorbox}[breakable, size=fbox, boxrule=1pt, pad at break*=1mm,colback=cellbackground, colframe=cellborder]
\prompt{In}{incolor}{58}{\boxspacing}
\begin{Verbatim}[commandchars=\\\{\}]
\PY{k+kn}{from} \PY{n+nn}{statsmodels}\PY{n+nn}{.}\PY{n+nn}{tsa}\PY{n+nn}{.}\PY{n+nn}{api} \PY{k+kn}{import} \PY{n}{ExponentialSmoothing}

\PY{c+c1}{\PYZsh{} Chargement des données de consommation dju}
\PY{n}{y} \PY{o}{=} \PY{n}{np}\PY{o}{.}\PY{n}{asarray}\PY{p}{(}\PY{n}{conso\PYZus{}dju}\PY{p}{[}\PY{l+s+s2}{\PYZdq{}}\PY{l+s+s2}{conso\PYZus{}correction}\PY{l+s+s2}{\PYZdq{}}\PY{p}{]}\PY{p}{)}

\PY{c+c1}{\PYZsh{} Création d\PYZsq{}un modèle de lissage exponentiel en utilisant les données de consommation}
\PY{c+c1}{\PYZsh{} La période saisonnière est définie à 71 mois et un trend et une saisonnalité sont ajoutés au modèle}
\PY{n}{hw} \PY{o}{=} \PY{n}{ExponentialSmoothing}\PY{p}{(}\PY{n}{y}\PY{p}{,} \PY{n}{seasonal\PYZus{}periods}\PY{o}{=}\PY{l+m+mi}{71}\PY{p}{,} \PY{n}{trend}\PY{o}{=}\PY{l+s+s1}{\PYZsq{}}\PY{l+s+s1}{add}\PY{l+s+s1}{\PYZsq{}}\PY{p}{,} \PY{n}{seasonal}\PY{o}{=}\PY{l+s+s1}{\PYZsq{}}\PY{l+s+s1}{add}\PY{l+s+s1}{\PYZsq{}}\PY{p}{)}\PY{o}{.}\PY{n}{fit}\PY{p}{(}\PY{p}{)}

\PY{c+c1}{\PYZsh{} Prédiction de la consommation pour les 12 prochains mois}
\PY{n}{hw\PYZus{}pred0} \PY{o}{=} \PY{n}{hw}\PY{o}{.}\PY{n}{forecast}\PY{p}{(}\PY{l+m+mi}{12}\PY{p}{)}

\PY{c+c1}{\PYZsh{} Affichage des prévisions}
\PY{n+nb}{print}\PY{p}{(}\PY{n}{hw\PYZus{}pred0}\PY{p}{)}
\end{Verbatim}
\end{tcolorbox}

    \begin{Verbatim}[commandchars=\\\{\}]
[32061.79135883 30355.71785545 31200.17037064 28306.5429269
 29971.88286687 31035.64330712 33002.66309132 31486.4728327
 32174.97077571 29529.24022186 29897.64965973 31184.2099622 ]
    \end{Verbatim}

    \begin{tcolorbox}[breakable, size=fbox, boxrule=1pt, pad at break*=1mm,colback=cellbackground, colframe=cellborder]
\prompt{In}{incolor}{59}{\boxspacing}
\begin{Verbatim}[commandchars=\\\{\}]
\PY{c+c1}{\PYZsh{} Visualisation de la prévision sur 12 mois par Holt\PYZhy{}Winters}
\PY{n}{figure}\PY{p}{(}\PY{n}{figsize}\PY{o}{=}\PY{p}{(}\PY{l+m+mi}{16}\PY{p}{,} \PY{l+m+mi}{5}\PY{p}{)}\PY{p}{)}
\PY{n}{plt}\PY{o}{.}\PY{n}{plot}\PY{p}{(}\PY{n}{conso\PYZus{}dju}\PY{p}{[}\PY{l+s+s2}{\PYZdq{}}\PY{l+s+s2}{conso\PYZus{}correction}\PY{l+s+s2}{\PYZdq{}}\PY{p}{]}\PY{p}{,} \PY{n}{label}\PY{o}{=}\PY{l+s+s1}{\PYZsq{}}\PY{l+s+s1}{Consommation corrigée}\PY{l+s+s1}{\PYZsq{}}\PY{p}{)}
\PY{n}{plt}\PY{o}{.}\PY{n}{plot}\PY{p}{(}\PY{n}{pd}\PY{o}{.}\PY{n}{date\PYZus{}range}\PY{p}{(}\PY{n}{conso\PYZus{}dju}\PY{o}{.}\PY{n}{index}\PY{p}{[}\PY{n+nb}{len}\PY{p}{(}\PY{n}{y}\PY{p}{)}\PY{o}{\PYZhy{}}\PY{l+m+mi}{1}\PY{p}{]}\PY{p}{,} \PY{n}{periods}\PY{o}{=}\PY{l+m+mi}{12}\PY{p}{,} \PY{n}{freq}\PY{o}{=}\PY{l+s+s1}{\PYZsq{}}\PY{l+s+s1}{M}\PY{l+s+s1}{\PYZsq{}}\PY{p}{)}\PY{p}{,} \PY{n}{hw\PYZus{}pred0}\PY{p}{,} \PY{n}{label}\PY{o}{=}\PY{l+s+s1}{\PYZsq{}}\PY{l+s+s1}{Prévision Holt\PYZhy{}Winters}\PY{l+s+s1}{\PYZsq{}}\PY{p}{)}
\PY{n}{plt}\PY{o}{.}\PY{n}{title}\PY{p}{(}\PY{l+s+s2}{\PYZdq{}}\PY{l+s+s2}{Consommation d}\PY{l+s+s2}{\PYZsq{}}\PY{l+s+s2}{électricité en France \PYZhy{} Prévision pour les 12 prochains mois}\PY{l+s+s2}{\PYZdq{}}\PY{p}{)}
\PY{n}{plt}\PY{o}{.}\PY{n}{xlabel}\PY{p}{(}\PY{l+s+s1}{\PYZsq{}}\PY{l+s+s1}{mois\PYZhy{}années}\PY{l+s+s1}{\PYZsq{}}\PY{p}{)}
\PY{n}{plt}\PY{o}{.}\PY{n}{ylabel}\PY{p}{(}\PY{l+s+s1}{\PYZsq{}}\PY{l+s+s1}{consommation d}\PY{l+s+se}{\PYZbs{}\PYZsq{}}\PY{l+s+s1}{électricité}\PY{l+s+s1}{\PYZsq{}}\PY{p}{)}
\PY{n}{plt}\PY{o}{.}\PY{n}{legend}\PY{p}{(}\PY{p}{)}
\PY{n}{plt}\PY{o}{.}\PY{n}{grid}\PY{p}{(}\PY{k+kc}{True}\PY{p}{)}
\PY{n}{plt}\PY{o}{.}\PY{n}{show}\PY{p}{(}\PY{p}{)}
\end{Verbatim}
\end{tcolorbox}

    \begin{center}
    \adjustimage{max size={0.9\linewidth}{0.9\paperheight}}{output_85_0.png}
    \end{center}
    { \hspace*{\fill} \\}
    
    \begin{tcolorbox}[breakable, size=fbox, boxrule=1pt, pad at break*=1mm,colback=cellbackground, colframe=cellborder]
\prompt{In}{incolor}{60}{\boxspacing}
\begin{Verbatim}[commandchars=\\\{\}]
\PY{n}{figure}\PY{p}{(}\PY{n}{figsize}\PY{o}{=}\PY{p}{(}\PY{l+m+mi}{16}\PY{p}{,} \PY{l+m+mi}{6}\PY{p}{)}\PY{p}{)}

\PY{n}{plt}\PY{o}{.}\PY{n}{plot}\PY{p}{(}\PY{n}{conso\PYZus{}dju}\PY{p}{[}\PY{l+s+s2}{\PYZdq{}}\PY{l+s+s2}{conso\PYZus{}correction}\PY{l+s+s2}{\PYZdq{}}\PY{p}{]}\PY{o}{.}\PY{n}{iloc}\PY{p}{[}\PY{o}{\PYZhy{}}\PY{l+m+mi}{12}\PY{p}{:}\PY{p}{]}\PY{p}{,} \PY{n}{marker}\PY{o}{=}\PY{l+s+s1}{\PYZsq{}}\PY{l+s+s1}{o}\PY{l+s+s1}{\PYZsq{}}\PY{p}{,} \PY{n}{color}\PY{o}{=}\PY{l+s+s1}{\PYZsq{}}\PY{l+s+s1}{blue}\PY{l+s+s1}{\PYZsq{}}\PY{p}{,} \PY{n}{label}\PY{o}{=}\PY{l+s+s1}{\PYZsq{}}\PY{l+s+s1}{Consommation corrigée}\PY{l+s+s1}{\PYZsq{}}\PY{p}{)}
\PY{n}{plt}\PY{o}{.}\PY{n}{plot}\PY{p}{(}\PY{n}{pd}\PY{o}{.}\PY{n}{date\PYZus{}range}\PY{p}{(}\PY{n}{conso\PYZus{}dju}\PY{o}{.}\PY{n}{index}\PY{p}{[}\PY{n+nb}{len}\PY{p}{(}\PY{n}{y}\PY{p}{)}\PY{o}{\PYZhy{}}\PY{l+m+mi}{1}\PY{p}{]}\PY{p}{,} \PY{n}{periods}\PY{o}{=}\PY{l+m+mi}{12}\PY{p}{,} \PY{n}{freq}\PY{o}{=}\PY{l+s+s1}{\PYZsq{}}\PY{l+s+s1}{M}\PY{l+s+s1}{\PYZsq{}}\PY{p}{)}\PY{p}{,} \PY{n}{hw\PYZus{}pred0}\PY{p}{,} \PY{n}{marker}\PY{o}{=}\PY{l+s+s1}{\PYZsq{}}\PY{l+s+s1}{x}\PY{l+s+s1}{\PYZsq{}}\PY{p}{,} \PY{n}{color}\PY{o}{=}\PY{l+s+s1}{\PYZsq{}}\PY{l+s+s1}{red}\PY{l+s+s1}{\PYZsq{}}\PY{p}{,} \PY{n}{label}\PY{o}{=}\PY{l+s+s1}{\PYZsq{}}\PY{l+s+s1}{Prévision Holt\PYZhy{}Winters}\PY{l+s+s1}{\PYZsq{}}\PY{p}{)}

\PY{n}{plt}\PY{o}{.}\PY{n}{title}\PY{p}{(}\PY{l+s+s2}{\PYZdq{}}\PY{l+s+s2}{Consommation d}\PY{l+s+se}{\PYZbs{}\PYZsq{}}\PY{l+s+s2}{électricité en France 2021 \PYZhy{} Prévision 2022}\PY{l+s+s2}{\PYZdq{}}\PY{p}{)}
\PY{n}{plt}\PY{o}{.}\PY{n}{xlabel}\PY{p}{(}\PY{l+s+s1}{\PYZsq{}}\PY{l+s+s1}{mois\PYZhy{}années}\PY{l+s+s1}{\PYZsq{}}\PY{p}{)}
\PY{n}{plt}\PY{o}{.}\PY{n}{ylabel}\PY{p}{(}\PY{l+s+s1}{\PYZsq{}}\PY{l+s+s1}{consommation d}\PY{l+s+se}{\PYZbs{}\PYZsq{}}\PY{l+s+s1}{électricité}\PY{l+s+s1}{\PYZsq{}}\PY{p}{)}
\PY{n}{plt}\PY{o}{.}\PY{n}{grid}\PY{p}{(}\PY{p}{)}
\PY{n}{plt}\PY{o}{.}\PY{n}{legend}\PY{p}{(}\PY{p}{)}


\PY{n}{plt}\PY{o}{.}\PY{n}{show}\PY{p}{(}\PY{p}{)}
\end{Verbatim}
\end{tcolorbox}

    \begin{center}
    \adjustimage{max size={0.9\linewidth}{0.9\paperheight}}{output_86_0.png}
    \end{center}
    { \hspace*{\fill} \\}
    
    En effet, la prévision à court terme de la consommation d'électricité
est un élément important pour assurer l'équilibre entre la production et
la consommation d'électricité au cours d'une journée. Cela permet aux
gestionnaires du réseau électrique de s'assurer qu'il y a suffisamment
de production pour répondre à la demande, et de prendre les mesures
nécessaires pour éviter des coupures de courant ou des surcharges sur le
réseau.

    \begin{tcolorbox}[breakable, size=fbox, boxrule=1pt, pad at break*=1mm,colback=cellbackground, colframe=cellborder]
\prompt{In}{incolor}{61}{\boxspacing}
\begin{Verbatim}[commandchars=\\\{\}]
\PY{n}{hw\PYZus{}pred} \PY{o}{=} \PY{n}{hw}\PY{o}{.}\PY{n}{forecast}\PY{p}{(}\PY{l+m+mi}{71}\PY{p}{)}\PY{c+c1}{\PYZsh{}prévoir sur 71 mois}
\PY{c+c1}{\PYZsh{}Visualisation de la prévision sur 71 mois par Holt\PYZhy{}Winters}
\PY{n}{figure}\PY{p}{(}\PY{n}{figsize}\PY{o}{=}\PY{p}{(}\PY{l+m+mi}{16}\PY{p}{,} \PY{l+m+mi}{6}\PY{p}{)}\PY{p}{)}
\PY{n}{plt}\PY{o}{.}\PY{n}{plot}\PY{p}{(}\PY{n}{conso\PYZus{}dju}\PY{p}{[}\PY{l+s+s2}{\PYZdq{}}\PY{l+s+s2}{conso\PYZus{}correction}\PY{l+s+s2}{\PYZdq{}}\PY{p}{]}\PY{p}{,} \PY{n}{marker}\PY{o}{=}\PY{l+s+s1}{\PYZsq{}}\PY{l+s+s1}{o}\PY{l+s+s1}{\PYZsq{}}\PY{p}{,} \PY{n}{color}\PY{o}{=}\PY{l+s+s1}{\PYZsq{}}\PY{l+s+s1}{blue}\PY{l+s+s1}{\PYZsq{}}\PY{p}{,} \PY{n}{label}\PY{o}{=}\PY{l+s+s1}{\PYZsq{}}\PY{l+s+s1}{Consommation corrigée}\PY{l+s+s1}{\PYZsq{}}\PY{p}{)}
\PY{n}{plt}\PY{o}{.}\PY{n}{plot}\PY{p}{(}\PY{n}{pd}\PY{o}{.}\PY{n}{date\PYZus{}range}\PY{p}{(}\PY{n}{conso\PYZus{}dju}\PY{o}{.}\PY{n}{index}\PY{p}{[}\PY{n+nb}{len}\PY{p}{(}\PY{n}{y}\PY{p}{)}\PY{o}{\PYZhy{}}\PY{l+m+mi}{1}\PY{p}{]}\PY{p}{,} \PY{n}{periods}\PY{o}{=}\PY{l+m+mi}{71}\PY{p}{,} \PY{n}{freq}\PY{o}{=}\PY{l+s+s1}{\PYZsq{}}\PY{l+s+s1}{M}\PY{l+s+s1}{\PYZsq{}}\PY{p}{)}\PY{p}{,} \PY{n}{hw\PYZus{}pred}\PY{p}{,} \PY{n}{marker}\PY{o}{=}\PY{l+s+s1}{\PYZsq{}}\PY{l+s+s1}{x}\PY{l+s+s1}{\PYZsq{}}\PY{p}{,} \PY{n}{color}\PY{o}{=}\PY{l+s+s1}{\PYZsq{}}\PY{l+s+s1}{red}\PY{l+s+s1}{\PYZsq{}}\PY{p}{,} \PY{n}{label}\PY{o}{=}\PY{l+s+s1}{\PYZsq{}}\PY{l+s+s1}{Prévision Holt\PYZhy{}Winters}\PY{l+s+s1}{\PYZsq{}}\PY{p}{)}
\PY{n}{plt}\PY{o}{.}\PY{n}{title}\PY{p}{(}\PY{l+s+s2}{\PYZdq{}}\PY{l+s+s2}{Consommation d}\PY{l+s+s2}{\PYZsq{}}\PY{l+s+s2}{électricité en France \PYZhy{} Prévision de 2022 à 2032}\PY{l+s+s2}{\PYZdq{}}\PY{p}{)}
\PY{n}{plt}\PY{o}{.}\PY{n}{legend}\PY{p}{(}\PY{p}{)}
\PY{n}{plt}\PY{o}{.}\PY{n}{grid}\PY{p}{(}\PY{k+kc}{True}\PY{p}{)}
\PY{n}{plt}\PY{o}{.}\PY{n}{show}\PY{p}{(}\PY{p}{)}
\end{Verbatim}
\end{tcolorbox}

    \begin{center}
    \adjustimage{max size={0.9\linewidth}{0.9\paperheight}}{output_88_0.png}
    \end{center}
    { \hspace*{\fill} \\}
    
    \begin{tcolorbox}[breakable, size=fbox, boxrule=1pt, pad at break*=1mm,colback=cellbackground, colframe=cellborder]
\prompt{In}{incolor}{62}{\boxspacing}
\begin{Verbatim}[commandchars=\\\{\}]
\PY{c+c1}{\PYZsh{}Visualisation de la prévision sur sur 142 mois par Holt\PYZhy{}Winters}
\PY{n}{figure}\PY{p}{(}\PY{n}{figsize}\PY{o}{=}\PY{p}{(}\PY{l+m+mi}{16}\PY{p}{,} \PY{l+m+mi}{6}\PY{p}{)}\PY{p}{)}
\PY{n}{plt}\PY{o}{.}\PY{n}{plot}\PY{p}{(}\PY{n}{conso\PYZus{}dju}\PY{p}{[}\PY{l+s+s2}{\PYZdq{}}\PY{l+s+s2}{conso\PYZus{}correction}\PY{l+s+s2}{\PYZdq{}}\PY{p}{]}\PY{p}{,} \PY{n}{label}\PY{o}{=}\PY{l+s+s1}{\PYZsq{}}\PY{l+s+s1}{Consommation corrigée}\PY{l+s+s1}{\PYZsq{}}\PY{p}{)}
\PY{n}{plt}\PY{o}{.}\PY{n}{plot}\PY{p}{(}\PY{n}{pd}\PY{o}{.}\PY{n}{date\PYZus{}range}\PY{p}{(}\PY{n}{conso\PYZus{}dju}\PY{o}{.}\PY{n}{index}\PY{p}{[}\PY{n+nb}{len}\PY{p}{(}\PY{n}{y}\PY{p}{)}\PY{o}{\PYZhy{}}\PY{l+m+mi}{1}\PY{p}{]}\PY{p}{,} \PY{n}{periods}\PY{o}{=}\PY{l+m+mi}{71}\PY{p}{,} \PY{n}{freq}\PY{o}{=}\PY{l+s+s1}{\PYZsq{}}\PY{l+s+s1}{2M}\PY{l+s+s1}{\PYZsq{}}\PY{p}{)}\PY{p}{,} \PY{n}{hw\PYZus{}pred}\PY{p}{,} \PY{n}{label}\PY{o}{=}\PY{l+s+s1}{\PYZsq{}}\PY{l+s+s1}{Prévision Holt\PYZhy{}Winters}\PY{l+s+s1}{\PYZsq{}}\PY{p}{)}
\PY{n}{plt}\PY{o}{.}\PY{n}{title}\PY{p}{(}\PY{l+s+s2}{\PYZdq{}}\PY{l+s+s2}{Consommation d}\PY{l+s+s2}{\PYZsq{}}\PY{l+s+s2}{électricité en France \PYZhy{} Prévision de 2022 à 2035}\PY{l+s+s2}{\PYZdq{}}\PY{p}{)}
\PY{n}{plt}\PY{o}{.}\PY{n}{xlabel}\PY{p}{(}\PY{l+s+s1}{\PYZsq{}}\PY{l+s+s1}{année}\PY{l+s+s1}{\PYZsq{}}\PY{p}{)}
\PY{n}{plt}\PY{o}{.}\PY{n}{ylabel}\PY{p}{(}\PY{l+s+s1}{\PYZsq{}}\PY{l+s+s1}{consommation d}\PY{l+s+se}{\PYZbs{}\PYZsq{}}\PY{l+s+s1}{électricité}\PY{l+s+s1}{\PYZsq{}}\PY{p}{)}
\PY{n}{plt}\PY{o}{.}\PY{n}{legend}\PY{p}{(}\PY{p}{)}
\PY{n}{plt}\PY{o}{.}\PY{n}{grid}\PY{p}{(}\PY{k+kc}{True}\PY{p}{)}
\PY{n}{plt}\PY{o}{.}\PY{n}{show}\PY{p}{(}\PY{p}{)}
\end{Verbatim}
\end{tcolorbox}

    \begin{center}
    \adjustimage{max size={0.9\linewidth}{0.9\paperheight}}{output_89_0.png}
    \end{center}
    { \hspace*{\fill} \\}
    
    \hypertarget{conclusion}{%
\section{Conclusion}\label{conclusion}}

En conclusion, l'utilisation des techniques de machine learning pour
analyser, classer et prédire la consommation d'électricité peut offrir
de nombreux avantages, tels que l'optimisation de la production
d'électricité et la réduction des coûts associés à la consommation
d'énergie. Grâce à leur capacité à extraire des informations précieuses
à partir de données complexes, les algorithmes d'apprentissage
automatique peuvent jouer un rôle crucial dans la gestion efficace de la
consommation d'électricité. Cependant, il est important de noter que
cette approche doit être utilisée avec prudence et en prenant en compte
les différents facteurs qui peuvent affecter la consommation
d'électricité, afin de garantir des résultats précis et fiables.


    % Add a bibliography block to the postdoc
    
    
    
\end{document}
